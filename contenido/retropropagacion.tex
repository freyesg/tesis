\section{Algoritmo de retropropagación}
Una regla de aprendizaje es el método que le permite adaptar los parámetros de la red. El perceptrón multicapa actualiza sus pesos en función de la salida obtenida de tal manera que los nuevos pesos permitan reducir el error de salida. Por tanto, para cada patrón de entrada a la red es necesario disponer de un patrón de salida deseada.

El objetivo es que la salida de la red sea lo más próxima posible a la salida deseada, debido a esto la es que el aprendizaje de la red se describe como un problema de minimización de la siguiente maner $$ \min_{W} E $$ donde $W$ es el conjunto de parámetros de la red (pesos y umbrales) y $E$ es una función de error que evalúa la diferencia entre las salidas de la red y las salidas deseadas. en la mayor parte de los casos, la función de error se define como:
\begin{eqnarray}
	E = \frac{1}{N}\sum^{N}_{i = 1} e(i)
\end{eqnarray}

Donde $N$ es el número de muestras y $e(n)$ es el error cometido por la red para el patrón $i$, definido de la siguiente manera
\begin{eqnarray}
	e(i) = \frac{1}{n_{C}}\sum^{n_{C}}_{j = 1} (s_{j}(i) - y^{j}(n))^2\label{eq:error_patron}
\end{eqnarray}

Siendo $Y(i) = (y_{1}(i), y_{2}(i), \cdots, y_{n_{C}}(i))$ y $S(i) = (s_{1}(i), s_{2}(i), \cdots, s_{n_{C}}(i))$ los vectores de salida y salidas deseadas para el patrón $i$ respectivamente.

De esta manera, si $W^{*}$ es un mínimo de la función de error $E$, en dicho punto el error será cercano a cero, y en consecuencia, la salida de la red será próxima a la salida deseada.

Así es como el aprendizaje es equivalente a encontrar un mínimo de la función de error. La presencia de funciones de activación no lineales hace que la respuesta de la red sea no lineal respecto a los parámetros ajustables, por lo que el problema de aminimización es un problema no lineal y se hace necesario el uso de técnicas de optización no lineales para su resolución.

Las técnicas utilizadas suelen basarse en la actualización de los parámetros de la red mediante la determinación de una dirección de búsqueda. En el caso de las redes neuronales multicapa, la dirección de búsqueda más utilizada se basa en la dirección contraria del gradiente de la función de error $E$, el método de gradiente descendente.

Si bien el aprendizaje de la red busca minimizar el error total de la red, el procedimiento está basado en métodos del agradiente estocástico, que son una sucesión de minimizaciones del error en función de cada patrón $e(i)$, en lugar de minimizar el error total $E$ de la red. Aplicando el método del gradiente estocástico, cada parámetro $w$ se modifica para cada patrón de entrada $n$ según la siguiente regla de aprendizaje
\begin{eqnarray}
	w(i) = w(n - 1) - \alpha\frac{\partial e(i)}{\partial w}
\end{eqnarray}

donde $e(i)$ es el error para el patrón de entrada $i$ dado por la ecuación \ref{eq:error_patron}, y $\alpha$ es la tasa de aprendizaje, éste último determina el desplazamiento en la superficie del error.

Como las neuronas están ordenadas por capas y en distintos niveles, es posible aplicar el método del gradiente de forma eficiente, resultando en el {\em algoritmo de retropropagación} \cite{Rumelhart1986} o {\em regla delta generalizada}. El término retropropagación es utilizado debido a la forma de implementar el método del gradiente en las redes multicapa, pues el error cometido en la salida de la red es propagado hacia atrás, transformándolo en un erro para cada una de las neuronas ocultas de la red.






% Neural Networks for Pattern Recognition - Bishop: 140 - Error backpropagation.
% Neural Networks for Pattern Recognition - Bishop: 263 - Gradient descent.
El algoritmo de retropropagación es el método de entrenamiento más utilizado en redes con conexión hacia adelante. Es un método de aprendizaje supervisado de gradiente descendente, en el que se distinguen claramente dos fases:
\begin{enumerate}
	\item Se aplica un patrón de entrada, el cual se propaga por las distintas capas que componen la red hasta producir la salida de la misma. Esta salida se compara con la salida deseada y se calcula el error cometido por cada neurona de salida.

	\item Estos errores se transmiten desde la capa de salida, hacia todas neuronas de las capas anteriores (Fritsch, 1996). Cada neurona recibe un error que es proporcional a su contribución sobre el error total de la red. Basándose en el error recibido, se ajustan los errores de los pesos sinápticos de cada neurona.
\end{enumerate}








%\section{Reglas de aprendizaje}
\section{El gradiente descendente}% http://alejandrosanchezyali.blogspot.cl/2016/01/algoritmo-del-gradiente-descendente-y.html
El gradiente descendente busca los punto $p \in \Omega$ donde funciones del tipo $f: \Omega\subseteq\mathbb{R}^m \rightarrow \mathbb{R}$ alcanzan su mínimo. La idea de este método se basa en que si $f$ es una función diferenciable en todo su dominio $\Omega$, entonces la derivada de $f$ es un punto $p \in \Omega$ en dirección de un vector unitario $v \in \mathbb{R}^m$ se define como

$$ df_{p}(v) = \nabla f(p)v $$

Observe que la magnitud de la ecuación es
$$ |d f_{p}(v)| = ||\nabla f(p)|| ||v||\cos\theta = ||\nabla f(p)\cos\theta$$

Dicha magnitud es máxima cuando $\theta = 2n\pi, n \in \mathbb{Z}$. Es decir, para que $|df_{d}(v)|$ sea máxima, los vectores $\nabla f(p)$ y $v$ debe ser paralelo. De esta manera, la función $f$ crece más rápidamente en la dirección del vector $\nabla f(p)$ y decrece más rápidamente en la dirección del vectro $-\nabla f(p)$. Dicha situación sugiere que la dirección negativa del gradiente $-\nabla f(p)$ es una buena dirección de búsqueda para encontrar el minimizador de la función $f$.

Sea $f: \Omega \subseteq \mathbb{R} \rightarrow \mathbb{R}$, si $f$ tiene un mínimo en $p$, para encontrar a $p$ se construye una sucesión de punto $\{p_{t}\}$ tal que $p_{t}$ converge a $p$. Para resolver esto, comenzamos en $p_{t}$ y nos desplazamos una cantidad $-\lambda_{t}\nabla f(p_{t})$ para encontrar el punto $p_{t + 1}$ más cercano a $p$, es decir:
$$ p_{t + 1} =p_{t} - \lambda _{t}\nabla f(p_{t}) $$

donde $\lambda_{t}$ se selecciona de tal manera que $p_{t + 1} \in \Omega$ y $f(p_{t}) \geq f(p_{t + 1})$

El parámetro $\lambda_{t}$ se seleccionara para maximizar la cantidad a la que decrece la función $f$ en cada paso.


\begin{figure}[H]
	\centering
    \scalebox{0.6}{% http://staff.itee.uq.edu.au/janetw/cmc/chapters/BackProp/index2.html
% http://outlace.com/Beginner-Tutorial-Backpropagation/
% http://neuralnetworksanddeeplearning.com/chap2.html
% http://staff.itee.uq.edu.au/janetw/cmc/chapters/BackProp/slides/Backprop_files/frame.htm
% http://staff.itee.uq.edu.au/janetw/cmc/chapters/BackProp/
% http://home.agh.edu.pl/~vlsi/AI/backp_t_en/backprop.html
% https://www.google.cl/webhp?sourceid=chrome-instant&ion=1&espv=2&ie=UTF-8#safe=off&q=backpropagation+practice+example
% https://ayearofai.com/rohan-4-the-vanishing-gradient-problem-ec68f76ffb9b#.9ntv81akz

\ifx\du\undefined
  \newlength{\du}
\fi
\newcommand{\nweigth}[2]{$W^{#1}_{#2} - \alpha\frac{\partial J}{\partial W^{#1}_{#2}}$}
\newcommand{\w}[2]{$W^{#1}_{#2}$}

\setlength{\du}{1\unitlength}
\begin{tikzpicture}[font=\small]
\tikzstyle{neuron}=[circle,draw, minimum size=2em]
\tikzstyle{update}=[dashed, blue]

\pgftransformxscale{1.000000}
\pgftransformyscale{-1.000000}

\coordinate (x1)    at (0.000000\du, 0.000000\du);
\coordinate (x2)    at (0.000000\du, 8.000000\du);

\coordinate (i)     at (3.000000\du, 0.000000\du);
\coordinate (ii)    at (3.000000\du, 8.000000\du);

\coordinate (iv)    at (10.000000\du, 0.000000\du);
\coordinate (v)     at (10.000000\du, 8.000000\du);

\coordinate (vii)   at (17.000000\du, 0.000000\du);
\coordinate (viii)  at (17.000000\du, 8.000000\du);

\coordinate (ix)   at (17.000000\du, 0.000000\du);
\coordinate (diez)  at (17.000000\du, 8.000000\du);


\coordinate (x)     at (24.000000\du, 4.000000\du);
\coordinate (y)     at (27.000000\du, 4.000000\du);


%%%%%%%%%%%%%%%%%%%%%%%%%%%%%%%%%%%%%%%%%%%%%%%%%%%%%%%%%%
% ENTRADA
\node (X1) at (x1) {\Huge $x_1$};
\node (X2) at (x2) {\Huge $x_2$};

% CAPA DE ENTRADA
\node[neuron] (A) at  (i) {}; % A
\node[neuron] (B) at  (ii)  {}; % B

% CAPA 1
\node[neuron] (C) at  (iv)  {}; % C
\node[neuron] (D) at  (v)  {}; % D

% CAPA 2
\node[neuron] (E) at  (vii)  {}; % E
\node[neuron] (F) at  (viii)  {}; % F

% CAPA 2
\node[neuron] (H) at  (ix)  {}; % E
\node[neuron] (I) at  (diez)  {}; % F

% CAPA DE SALIDA
\node[neuron] (G) at  (x)  {}; % G

% SALIDA
\node (out) at (y) {\Huge $y_1$};
%%%%%%%%%%%%%%%%%%%%%%%%%%%%%%%%%%%%%%%%%%%%%%%%%%%%%%%%%%


\draw[->] (X1) -- (A);
\draw[->] (X2) -- (B);


%%%%%%%%%%%%%%%%%%%%%%%%%%%%%%%%%%%%%%%%%%%%%%%%%%%%%%%%%%
%%%%%%%%%%%%%%%%%%%%%%%%%%%%%%%%%%%%%%%%%%%%%%%%%%%%%%%%%%
\draw[->] (A) -- node[above, pos=0.2] (W_1_11) {\w{1}{11}} (C);
\draw[->] (A) -- node[left , pos=0.2] (W_1_12) {\w{1}{12}} (D);
\draw[->] (B) -- node[left , pos=0.2] (W_1_13) {\w{1}{13}} (C);
\draw[->] (B) -- node[below, pos=0.2] (W_1_14) {\w{1}{14}} (D);

%\draw[->, dashed, blue] (C) to [bend left=40] node[above] {$W^{1}_{11} - \alpha\frac{\partial J}{\partial W^{1}_{11}}$} (W_1_11);
\draw[->, update] (C) to [bend left=40] node[above] {\nweigth{1}{11}} (W_1_11);
\draw[->, update] (D) to [bend left=50] node[above right, pos=0.98] {\nweigth{1}{12}} (W_1_12);

\draw[->, update] (C) to [bend right=50] node[below right, pos=0.98] {\nweigth{1}{13}} (W_1_13);
\draw[->, update] (D) to [bend right=40] node[below] {\nweigth{1}{14}} (W_1_14);

%%%%%%%%%%%%%%%%%%%%%%%%%%%%%%%%%%%%%%%%%%%%%%%%%%%%%%%%%%
%%%%%%%%%%%%%%%%%%%%%%%%%%%%%%%%%%%%%%%%%%%%%%%%%%%%%%%%%%
\draw[->] (C) -- node[above, pos=0.2] (W_2_11) {\w{l}{11}} (E);
\draw[->] (C) -- node[left , pos=0.2] (W_2_12) {\w{2}{12}} (F);
\draw[->] (D) -- node[left , pos=0.2] (W_2_13) {\w{2}{13}} (E);
\draw[->] (D) -- node[below, pos=0.2] (W_2_14) {\w{2}{14}} (F);

\draw[->, update] (E) to [bend left=40] node[above] {\nweigth{2}{11}} (W_2_11);
\draw[->, update] (F) to [bend left=50] node[above right, pos=0.98] {\nweigth{2}{12}} (W_2_12);

\draw[->, update] (E) to [bend right=50] node[below right, pos=0.98] {\nweigth{2}{13}} (W_2_13);
\draw[->, update] (F) to [bend right=40] node[below] {\nweigth{2}{14}} (W_2_14);

%%%%%%%%%%%%%%%%%%%%%%%%%%%%%%%%%%%%%%%%%%%%%%%%%%%%%%%%%%
%%%%%%%%%%%%%%%%%%%%%%%%%%%%%%%%%%%%%%%%%%%%%%%%%%%%%%%%%%
\draw[->] (E) -- node[right, pos=0.2] (W_3_11) {\w{3}{11}} (G);
\draw[->] (F) -- node[right, pos=0.2] (W_3_12) {\w{3}{12}} (G);

\draw[->, update] (G) to [bend  left=50] node[above right, pos=0.98] {\nweigth{3}{11}} (W_3_11);
\draw[->, update] (G) to [bend right=40] node[below right, pos=0.98] {\nweigth{3}{12}} (W_3_12);


\draw[->] (G) -- (out);
\end{tikzpicture}
}
    \caption{$W^{1}_{ij}$ es el peso de la $n$-ésima neurona en la capa $l - 1$ a la $j$-ésima neurona de la capa $l$ de la red.}
\end{figure}

\subsection{El desvanecimiento del gradiente}
% http://neuralnetworksanddeeplearning.com/chap5.html
