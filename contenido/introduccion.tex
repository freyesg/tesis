%%%%%%%%%%%%%%%%%%%%%%%%%%%%%%%%%%%%%%%%%%%%%%%%%%%%%%%%%%%%%%
% https://es.sharelatex.com/learn/Pgfplots_package#/Plotting_mathematical_expressions
%%%%%%%%%%%%%%%%%%%%%%%%%%%%%%%%%%%%%%%%%%%%%%%%%%%%%%%%%%%%%%


\chapter{Introducción}
%\subsection{Redes neuronales}
\citeA{McCulloch1943} presentaron el primer modelo de redes neuronales artificiales ({\em Neural networks}, NN) en términos de un modelo computacional de actividad nerviosa. Las NN se han inspirado en las redes neuronales biológicas y las conexiones que construyen. Este modelo era un modelo binario, donde cada neurona posee un umbral, y sirvió de base para los modelos posteriores.

Las características principales de las NN son las siguientes:
\begin{enumerate}
	\item Auto-organización y adaptabilidad: utilizan algoritmos de aprendizaje adaptativo y auto-organizativo, por lo que ofrecen mejores posibilidades de procesado robusto y adaptativo.

	\item Procesado no lineal: aumenta la capacidad de la red para aproximar funciones, clasificar patrones y aumenta su inmunidad frente al ruido.

	\item Procesado paralelo: normalmente se usa un gran número de nodos de procesado, con alto nivel de interconectividad.
\end{enumerate}

\section{Redes neuronales}
Una NN relaciona un conjunto de entradas $\{x_{i}\}, i = 1, 2, \cdots, k$, con un conjunto de una o más variables de salida $\{y_j\}, j = 1, 2, \cdots, k^{*}$. Utilizan una o más capas ocultas, en donde las variables de entrada son transformadas por una función conocida como funcion de activación.



