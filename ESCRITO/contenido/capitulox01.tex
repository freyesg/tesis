\chapter{Introducción}
\section{Antecedentes y motivación}

%%%%%%%%%%%%%%%%%%%%%%%%%%%%%%%%%%%%%%%%%%%%%%%%%%%%%
%%%%%%%%%%%%%%%%%%%%%%%%%%%%%%%%%%%%%%%%%%%%%%%%%%%%%
%%%%%%%%%%%%%%%% lista1
Las redes neuronales artificiales ({\em Artificial Neural Networks}, NN) han sido ampliamente estudiadas y ampliamente utilizadas en muchas aplicaciones de la inteligencia artificial. El problema durante el proceso de aprendizaje de las NN es descrito como un problema de minimización de una función de error, la que depende de los pesos que conforman la red \cite{Rumelhart1986}. Este problema de optimización tiene la desventaja de ser no lineal, no convexo, además de tener mas de un mínimo local. Para solventar este problema se han desarrollado diversos algoritmos \cite{Grippo1994,Jacobs1988,Plagianakos2002,Rumelhart1986b,Plagianakos1998}  y su rendimiento varía según el problema a resolver.

El enfoque clásico para el entrenamiento de las NN es la aplicación de algoritmos basados en el gradiente como la retropropagación \cite{Rumelhart1986b}. El algoritmo de retropropagación busca minimizar la función de error mediante la dirección de descenso más pronunciada. Aunque la función de error disminuye rápidamente en la dirección del gradiente negativo, la retropropagación es generalmente ineficiente y poco fiable \cite{Gori1992} debido a la superficie de error. Además, su rendimiento se ve afectado por parámetros que deben ser especificados por el usuario, pues no existe una base teórica para escogerlos \cite{Nguyen1990}. Dichos parámetros tienen una importancia crucial en el buen funcionamiento del algoritmo, por lo que el diseñador está obligado a seleccionar parámetros como los pesos iniciales de la NN, la topología de la red y la tasa de aprendizaje. En diversas investigaciones \cite{Cauchy1847, Grippo1994, Plagianakos1998, Plagianakos2002} ha quedado demostrado que pequeñas modificaciones en estos valores influyen en el rendimiento de la NN.

Para proporcionar una convergencia más rápida y estable se han desarrollado diversas variaciones y alternativas a la retroprogación. Algunos de estos métodos son la adaptación de un término de momento \cite{Jacobs1988, Rumelhart1986b} o de una tasa variable de aprendizaje \cite{Jacobs1988,Vogl1988}. \citeA{Magoulas1997,  Plagianakos1998} propusieron dos técnicas para evaluar en forma dinámica la tasa de aprendizaje sin el uso de alguna heurística o alguna función adicional y las evaluaciones de gradiente. El primero se basó en el algoritmo de Barzilai y Borwein \cite{Barzilai1988} que adapta la tasa de aprendizaje sin evaluar la matriz Hessiana; mientras que el segundo utiliza estimaciones de la constante de Lipschitz, explotando la información local de la superficie de error y los pesos posteriores \cite{Magoulas1997}.% Hay evidencias \cite{Magoulas1997, Plagianakos2002, Plagianakos1998} que han demostrado que la retropropagación con algoritmos que adaptan la velocidad del aprendizaje son robustas y tienen un buen rendimiento para el entrenamiento de NN.

Se han sugerido métodos de segundo orden  para mejorar la eficiencia del proceso de minimización del error. Algunos de los métodos utilizados son el del gradiente conjugado \cite{Fletcher1964, Hestenes1952, Polak1969} y el quasi-Newton \cite{Huang1970, Nocedal2006}. Los métodos del gradiente conjugado utiliza una combinación lineal de la dirección de búsqueda anterior y elgradiente actual lo que produce una convergencia generalmente más rápida, es adecuado para redes neuronales de gran escala debido a su simplicidad, sus propiedades de convergencia y la poca memoria que requiere. En la literatura  se encuentran diversos métodos basados en el gradiente conjugado \cite{Birgin2001, Moller1993} que han sido utilizados para la construcción de NN en varias aplicaciones \cite{Charalambous1992, Peng2007, Sotiropoulos2002}. Los métodos quasi-Newton se consideran como los algoritmos más sofisticados para el entrenamiendo rápido de una NN. Definen la dirección de búsqueda mediante una aproximación de la matriz Hessiana, requiriendo información adicional. Se han propuesto soluciones para ajustar la aproximación Hessiana mediante la introducción de distintas estrategias \cite{Al-Baali1998, Nocedal1993, Oren1972, Oren1974, Yin2007}. Estas estrategias combinadas con búsquedas de lineas no monótonas permitieron definir una convergencia superlineal y global \cite{Yin2007}, mejorando significativamente el rendimiento de los métodos originales.
%%%%%%%%%%%%%%%%%%%%%%%%%%%%%%%%%%%%%%%%%%%%%%%%%%%%%
%%%%%%%%%%%%%%%%%%%%%%%%%%%%%%%%%%%%%%%%%%%%%%%%%%%%%


%%%%%%%%%%%%%%%%%%%%%%%%%%%%%%%%%%%%%%%%%%%%%%%%%%%%%
%%%%%%%%%%%%%%%%%%%%%%%%%%%%%%%%%%%%%%%%%%%%%%%%%%%%%
%%%%%%%%%%%%%%%% lista1
Las NN han sido testigos de un renacimiento en el campo de las máquinas de aprendizaje ({\em Machine learning}, ML) con el surgimiento del {\em aprendizaje profundo} \cite{Bengio2006, Hinton2006, Le2012, Ranzato2007}. Las principales ideas detrás del nuevo enfoque abarcan una gama de algoritmos \cite{Bengio2007, Hinton2006}, pero un principio unificador es que una NN con múltiples capas ocultas (que lo hacen profundo) puede codificar características cada vez más complejas en sus capas superior. Las NN fueron entrenadas históricamente a través del algoritmo de retropropagación \cite{Rumelhart1986b}, que aplica el gradiente estocástico descendente ({\em Stochastics descent gradiente}, SGD) o una de sus variantes a los pesos de la NN para reducir el error total, y no fue hasta el año 2006 que fue difundido que el gradiente el método de la retropropagación disminuiría el valor del gradiente en las redes profundas. Sin embargo, los descubrimientos en los últimos años han demostrado que, con suficientes datos de entrenamiento y poder de procesamiento, el método de retropropagación y SGD resultan ser eficaces en la optimización de una NN masiva con millones de conexiones y muchas capas \cite{Ciresan2012, He2015, Le2012}. Esta realización ha llevado a registros sustantivos que se rompen en muchas áreas de las ML a través de la aplicación de la retropropagación en el aprendizaje profundo \cite{Ciresan2012, He2015, Le2012}, incluyendo el aprendizaje no supervisado \cite{Bengio2009}.
%%%%%%%%%%%%%%%%%%%%%%%%%%%%%%%%%%%%%%%%%%%%%%%%%%%%%
%%%%%%%%%%%%%%%%%%%%%%%%%%%%%%%%%%%%%%%%%%%%%%%%%%%%%






\section{Descripción del problema}
% USAR [2003b] Squartini S, Hussain A, Piazza F
La retropropagación basa su funcionamiento en multiplicaciones sucesivas basadas en el error para poder calcular los gradientes, y a medida que el error se propaga hacia la capa de entrada de la red él gradiente comienza a disminuír su valor por cada capa que atraviesa. Esto significa que el gradiente disminuirá de manera exponencial, lo que representa un problema para redes profundas, ya que las capas mas cercanas a la capa de entrada necesitarán más tiempo para ser entrenadas.

El método de aprendizaje basado en simmulated annealing permite la actualización de los pesos de la red sin mermar la capacidad de adaptación de los pesos. El método supone una alternativa efectiva a los métodos tradicionales de aprendizaje para la convergencia de los métodos debido a la independencia que otorga a la actualización de los pesos de las distintas capas..


\section{Solución propuesta}
\subsection{Características de la solución}
Mediante el uso de el algoritmo {\em simulated annealing} se busca analizar la eficiencia que la NN alcanza en una red neuronal profunda frente a otros métodos de aprendizaje tales como SGD y RMSPROP.

\subsection{Propósito de la solución}
El propósito de la solución es aportar en el campo de las redes neuronales y la clasificación de datos, proporcionando un análisis comparativo de la convergencia de distintas redes.

\section{Objetivos y alcances del proyecto}
En ésta sección se presenta el objetivo general, los objetivos específicos además del alcance y limitaciones de la presenta investigación.

\subsection{Objetivo general}
Evaluar el desempeño del algoritmo {\em simulated annealing} y su efecto sobre el entrenamiento de redes neuronales profundas en comparación con otros métodos.

\subsection{Objetivos específicos}
Los objetivos establecidos para el presente trabajo son descritos a continuación
\begin{enumerate}
	\item Definir las reglas de aprendizaje a comparar.
	\item Construir los conjuntos de datos de entrada y salida a analizar.
	\item Establecer los parámetros de las redes neuronales para la experimentación.
	\item Establecer los algoritmos de aprendizaje a comparar.
	\item Entrenar las redes con los distintos conjuntos de datos.
	\item Establecer las conclusiones del trabajo.
\end{enumerate}

\subsection{Alcances}
\begin{enumerate}
	\item Se analizará la misma arquitectura con diferentes reglas de aprendizaje.
	\item Los conjunto de datos para el entrenamiento a utilizar son los propuestos en \cite{Morse2016}.
\end{enumerate}

\section{Metodología y herramientas utilizadas}
\subsection{Metodología de trabajo}
Considerando el aspecto investigativo del trabajo, se considera la utilización del método científico. Entre las actividades que componen la metodología, \citeA{Sampieri2006} describe los siguientes pasos para desarrollar una investigación:

\begin{itemize}
	\item Formulación de la hipótesis: Las redes neuronales que adolecen del desvanecimiento del gradiente se ven beneficiadas por el uso del algoritmo {\em simulated annealing} en la convergencia.

	\item Marco teórico: Una revisión de la literatura donde se aborda el problema planteado, para situarse en el contexto actual de los problemas. Se describirán redes neuronales que buscan solucionar el mismo problema.

	\item Diseño de la solución: Se deberá diseñar el experimento para generar los datos que permitan sustentar las comparaciones entre las distintas redes.% Diseñar y ejecutar el experimento basado en entradas equivalentes.

	\item Análisis y verificación de los resultados: Los resultados se analizarán considerando los valores de convergencia de los distintos métodos.

	\item Presentación de los resultados: Se presentarán tablas que describan los resultados obtenidos y que se consideren pertinentes.

	\item Conclusiones obtenidas en el desarrollo de la investigación.
\end{itemize}

\subsection{Herramientas de desarrollo}
Para el desarrollo y ejecución de los experimentos se utilizará un equipo con las siguientes características
\begin{table}[H]
	\centering
	\begin{tabular}{|l|l|}\hline
		Sistema Operativo	& Solus 2017.04.18.0 64-bit\\\hline
		Procesador				 & Intel$^\circledR$ Core\texttrademark i5-2450M CPU @ 2.50GHz x 4\\\hline
		RAM							  & 7.7Gb\\\hline
		Gráficos					& Intel$^\circledR$ Sandybridge Mobile\\\hline
		Almacenamiento	   & 935.6 GB\\\hline
	\end{tabular}
	\caption{Especificaciones del equipo}
\end{table}

El software que se utilizará es:
\begin{itemize}
	%\item Plataforma de desarrollo: Atom.
	\item Lenguaje de programación: Python.
	\item Sistema de redes neuronales: Keras API \cite{Keras2015}.
	\item Herramienta ofimática: \LaTeX.
\end{itemize}
