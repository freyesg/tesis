\chapter{Introducción}
\section{Antecedentes y motivación}
%La clasificación de patrones es una tarea que ha sido desarrollada por largo tiempo a través de numerosos métodos. Algunos precisan del conocimiento de la salida esperada (REFERENCIAS, REFERENCIAS, aprendizaje supervisado), mientras que otros métodos clasifican en forma automática las entradas según un criterio definido para el algoritmo en cuestion (REFERENCIAS, REFERENCIAS, aprendizaje no supervisado)
Las redes neuronales artificiales ({\em Artificial Neural Networks}, NN) han sido ampliamente estudiadas y ampliamente utilizadas en muchas aplicaciones de la inteligencia artificial. El problema durante proceso de aprendizaje de las NN es descrito como un problema de minimización de una función de error, la que depende de los pesos que conforman la red \cite{Rumelhart1986}. Este problema de optimización tiene la desventaja de ser no lineal, no convexo, además de tener mas de un mínimo local. Para solventar este problema se han desarrollado diversos algoritmos \cite{Grippo1994,Jacobs1988,Plagianakos2002,Rumelhart1986b,Plagianakos1998}  y su rendimiento varía según el problema a resolver.

El enfoque clásico para el entrenamiento de las NN es la aplicación de algoritmos basados en el gradiente como la retropropagación \cite{Rumelhart1986b}. El algoritmo de retropropagación minimiza la función de error mediante la dirección de descenso más pronunciada. Aunque la función de error disminuye rápidamente en la dirección del gradiente negativo, la retropropagación es generalmente ineficiente y poco fiable \cite{Gori1992} debido a la superficie de error. Además, su rendimiento se ve afectado por parámetros que deben ser especificados por el usuario, pues no existe una base teórica para escogerlos \cite{Nguyen1990}. Dichos parámetros tienen una importancia crucial en el buen funcionamiento del algoritmo, por lo que el diseñador está obligado a seleccionar parámetros como los pesos iniciales de la NN, la topología de la red y la tasa de aprendizaje. En diversas investigaciones \cite{Cauchy1847,Grippo1994,Plagianakos1998,Plagianakos2002} ha quedado demostrado que pequeñas modificaciones en estos valores influyen en el rendimiento de la NN. Para proporcionar una convergencia más rápida y estable se han desarrollado diversas variaciones y alternativas a la retroprogación.

% [Cauchy1847, Grippo1994, Plagianakos2002, Plagianakos1998]
%10 - - A. Cauchy. Methode generale pour la resulution des systemes d’equations simultanees. Comp. Rend. Acad. Sci. Paris, pages 536–538, 1847.
%19 - - L. Grippo. A class of unconstrained minimization methods for neural network training. Optimization Methods and Software, 4:135–150, 1994.
%40 - - V.P. Plagianakos, G.D. Magoulas, and M.N. Vrahatis. Determing nonmonotone strategies for effective training of multi-layer perceptrons. IEEE Transactions on Neural Networks, 13(6):1268–1284, 2002.
%41 - - V.P. Plagianakos, D.G. Sotiropoulos, and M.N. Vrahatis. Automatic adaptation of learning rate for backpropagation neural networks. In N.E. Mastorakis, editor, Recent Advantages in Circuits and Systems, pages 337–341, Singapure, 1998.
%%%%%%%%%%%%%%%%%%%%%%%%%%%
%En la literatura de redes neuronales un enfoque clásico para el entrenamiento de una red neuronal es La aplicación de algoritmos basados ​​en gradiente como la retropropagación que fue introducido por Rumelhart et al. [49]. En particular, el algoritmo de retropropagación minimiza la función de error utilizando la dirección de descenso más pronunciada. Aunque la función de error está disminuyendo rápidamente a lo largo de la dirección del gradiente negativo, desafortunadamente backpropagation es generalmente muy ineficiente y poco fiable [18] debido a la morfología de la superficie de error.
%Además, esto también se debe a que su rendimiento depende también de parámetros que han de ser heurísticamente especificados por el usuario, porque no existe una base teórica para elegirlos [30]. Los valores de estos parámetros son a menudo cruciales para el éxito del algoritmo, por lo que el diseñador está obligado a seleccionar arbitrariamente parámetros tales como pesos iniciales, topología de red y un valor de tasa de aprendizaje.
%Se ha establecido por muchos investigadores [10, 19, 40, 41] que variaciones muy pequeñas en estos valores pueden hacer la diferencia entre el rendimiento bueno, medio o malo y la elección apropiada de ellos constituye uno de los temas de mayor preocupación en la Literatura [19].
%Para proporcionar una convergencia más rápida y estable se han desarrollado varias variaciones y modificaciones de la retropropagación.





\section{Descripción del problema}
% USAR [2003b] Squartini S, Hussain A, Piazza F
La retropropagación basa su funcionamiento en multiplicaciones sucesivas basadas en el error para poder calcular los gradientes, y a medida que el error se propaga hacia la capa de entrada de la red él gradiente comienza a disminuír su valor por cada capa que atraviesa. Esto significa que el gradiente disminuirá de manera exponencial, lo que representa un problema para redes profundas, ya que las capas mas cercanas a la capa de entrada necesitarán más tiempo para ser entrenadas.

El método de aprendizaje basado en simmulated annealing permite la actualización de los pesos de la red sin mermar la capacidad de adaptación de los pesos. El método supone una alternativa efectiva a los métodos tradicionales de aprendizaje para la convergencia de los métodos debido a la independencia que otorga a la actualización de los pesos de las distintas capas..


\section{Solución propuesta}
\subsection{Características de la solución}
Mediante el uso de el algoritmo {\em simulated annealing} se busca analizar la eficiencia que la NN alcanza en una red neuronal profunda frente a otros métodos de aprendizaje tales como SGD y RMSPROP.

\subsection{Propósito de la solución}
El propósito de la solución es aportar en el campo de las redes neuronales y la clasificación de datos, proporcionando un análisis comparativo de la convergencia de distintas redes.

\section{Objetivos y alcances del proyecto}
En ésta sección se presenta el objetivo general, los objetivos específicos además del alcance y limitaciones de la presenta investigación.

\subsection{Objetivo general}
Evaluar el desempeño del algoritmo {\em simulated annealing} y su efecto sobre el entrenamiento de redes neuronales profundas en comparación con otros métodos.

\subsection{Objetivos específicos}
Los objetivos establecidos para el presente trabajo son descritos a continuación
\begin{enumerate}
	\item Definir las reglas de aprendizaje a comparar.
	\item Construir los conjuntos de datos de entrada y salida a analizar.
	\item Establecer los parámetros de las redes neuronales para la experimentación.
	\item Establecer los algoritmos de aprendizaje a comparar.
	\item Entrenar las redes con los distintos conjuntos de datos.
	\item Establecer las conclusiones del trabajo.
\end{enumerate}

\subsection{Alcances}
\begin{enumerate}
	\item Se analizará la misma arquitectura con diferentes reglas de aprendizaje.
	\item Los conjunto de datos para el entrenamiento a utilizar son los propuestos en \cite{Morse2016}.
\end{enumerate}

\section{Metodología y herramientas utilizadas}
\subsection{Metodología de trabajo}
Considerando el aspecto investigativo del trabajo, se considera la utilización del método científico. Entre las actividades que componen la metodología, \citeA{Sampieri2006} describe los siguientes pasos para desarrollar una investigación:

\begin{itemize}
	\item Formulación de la hipótesis: Las redes neuronales que adolecen del desvanecimiento del gradiente se ven beneficiadas por el uso del algoritmo {\em simulated annealing} en la convergencia.

	\item Marco teórico: Una revisión de la literatura donde se aborda el problema planteado, para situarse en el contexto actual de los problemas. Se describirán redes neuronales que buscan solucionar el mismo problema.

	\item Diseño de la solución: Se deberá diseñar el experimento para generar los datos que permitan sustentar las comparaciones entre las distintas redes.% Diseñar y ejecutar el experimento basado en entradas equivalentes.

	\item Análisis y verificación de los resultados: Los resultados se analizarán considerando los valores de convergencia de los distintos métodos.

	\item Presentación de los resultados: Se presentarán tablas que describan los resultados obtenidos y que se consideren pertinentes.

	\item Conclusiones obtenidas en el desarrollo de la investigación.
\end{itemize}

\subsection{Herramientas de desarrollo}
Para el desarrollo y ejecución de los experimentos se utilizará un equipo con las siguientes características
\begin{table}[H]
	\centering
	\begin{tabular}{|l|l|}\hline
		Sistema Operativo	& Solus 2017.04.18.0 64-bit\\\hline
		Procesador				 & Intel$^\circledR$ Core\texttrademark i5-2450M CPU @ 2.50GHz x 4\\\hline
		RAM							  & 7.7Gb\\\hline
		Gráficos					& Intel$^\circledR$ Sandybridge Mobile\\\hline
		Almacenamiento	   & 935.6 GB\\\hline
	\end{tabular}
	\caption{Especificaciones del equipo}
\end{table}

El software que se utilizará es:
\begin{itemize}
	%\item Plataforma de desarrollo: Atom.
	\item Lenguaje de programación: Python.
	\item Sistema de redes neuronales: Keras API \cite{Keras2015}.
	\item Herramienta ofimática: \LaTeX.
\end{itemize}
