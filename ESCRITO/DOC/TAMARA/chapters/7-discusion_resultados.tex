
\chapter{DISCUSIÓN DE LOS RESULTADOS}
\label{cap:discusion_resultados}

%DESCOMENTAR ESTAS LÍNEAS SI EL CAPÍTULO TIENE FIGURAS O TABLAS
% \addtocontents{lof}{{\bf \noindent Figuras del capítulo \arabic{chapter}}}
% \addtocontents{lot}{{\bf \noindent Tablas del capítulo \arabic{chapter}}}

Durante el proceso evolutivo fueron generados cientos de miles de algoritmos por cada experimento. De estos algoritmos, se escogieron solo los 5 mejores por cada experimento siendo analizados en total 70 de ellos (60 para el PM-01 y 10 para el PVV). A partir de estos algoritmos seleccionados y analizados, se procedió a encontrar a los mejores de cada experimento para realizar un análisis de la estructura de éstos. Los mejores de cada experimento fueron seleccionados en base a dos criterios: el algoritmo que tuviese el menor ERP y el algoritmo que entregase la mayor cantidad de soluciones óptimas. Esto último no se consiguió en el problema del vendedor viajero, ya que ninguno de los algoritmos encontrados pudo obtener soluciones óptimas de las instancias seleccionadas. Por lo que finalmente, para los experimentos 1 y 2 fueron seleccionados los cuatro mejores algoritmos (dos por menor ERP y dos por mayor cantidad de soluciones óptimas) y para los experimentos 3 y 4 los dos mejores de acuerdo al menor ERP obtenido.

Las estructuras encontradas dentro de los mejores algoritmos poseen características similares, por lo que es posible considerar que éstas fueron las más exitosas dentro de la evolución. Para el problema de la mochila bidimensional, la estructura con mayor éxito fue la que contenía los terminales “AgregarMayorGanancia”, “AgregarMenosPeso” y “EliminarPeorGanancia”, donde éstos operan para realizar el llenado de la mochila por medio del terminal “AgregarMayorGanancia”, la que agrega ítems a la mochila de acuerdo a la mayor ganancia (relación $beneficio/peso$), realizando un refinamiento con “EliminarPeorGanancia” que es el inverso del anterior, donde se elimina el de peor ganancia y, posteriormente, utilizar el terminal que agrega los elementos con menor peso “AgregarMenosPeso”. Estos terminales son iterados mediante el uso de las funciones, dentro de las que destacan “Do\_While” e “IfThen”. Los resultados obtenidos por estos algoritmos no superan el $1\%$ de error para los grupos de instancias utilizadas en el conjunto de evaluación en comparación al resultado óptimo de la literatura. En relación a los algoritmos que fueron obtenidos por los experimentos 3 y 4 (para el problema del vendedor viajero), también existen tres terminales que destacan dentro de la estructura “AMenorArco”, “AMejorVecino” y “ACercano”. Estos terminales siguen una estructura que busca armar el circuito de forma completa, sin realizar cambios eliminando parte del circuito. La forma en que estos terminales opera se basa, principalmente, en agregar el circuito que genere el arco de menor costo utilizando el terminal “AMenorArco”. Los otros terminales (“AMejorVecino” y “ACercano) operan para obtener un pequeño tramo del circuito, el que da paso para completar el circuito con el terminal “AMenorArco”. Al igual que los algoritmos obtenidos en los experimentos 1 y 2, éstos operan utilizando principalmente las mismas funciones (“Do\_While” e “IfThen”). Los resultados obtenidos por los algoritmos para el PVV promedian el $7\%$ de error para los grupos de instancias utilizadas en el conjunto de evaluación en comparación al resultado óptimo de la literatura.

Los mejores algoritmos generados por los métodos utilizados poseen un tamaño (cantidad de nodos) reducido, siendo los más grandes de 15 nodos. Este tamaño fue el determinado como número máximo de nodos permitido y todos los mejores algoritmos tienden a lo más, a este tamaño. El tamaño permite realizar un análisis sobre la estructura que éstos poseen con el fin de identificar el funcionamiento y sus principales características. Este análisis de la estructura permite identificar a los algoritmos como heurísticas constructivas y con algún tipo de refinamiento (en el caso de los relacionados al PM-01) y solo constructivas (para los relacionados al PVV).

La sobre-especialización de los algoritmos obtenidos mediante ambos métodos pudo ser constatada para los algoritmos obtenidos por los experimentos 1 y 2 (PM-01). Dada la gran cantidad de instancias y las clasificaciones realizadas para éstas en la literatura \citep{pisinger_2005}, se pudo realizar gran variedad de sub-experimentos, los que permitieron dar a conocer para el conjunto de terminales y funciones definidos, las instancias más fáciles y más difíciles de resolver; además, los algoritmos tienden a sobre-especializarse cuando todas las instancias utilizadas en el proceso evolutivo poseen las mismas características. Otro factor que pudo ser constatado gracias a la clasificación de instancias, fue que los algoritmos durante el proceso evolutivo obtienen mejores resultados para instancias de mayor tamaño.

Para los experimentos 2 y 4 se utilizaron diferentes funciones de evaluación (por HIT y por ERP). Al analizar la calidad que tienen los algoritmos generados en base a su ERP en comparación a su función de evaluación, fue posible determinar que en los casos donde se utiliza la función de evaluación por HIT, para algunos casos el ERP era menor que el obtenido por las que utilizan la función de evaluación con ERP. Esto se debe a que los algoritmos evaluados con la función por HIT, deben cumplir con un error relativo máximo para que sea considerado, resultando en que a pesar de que el ERP sea bajo, el valor de su función de evaluación no lo es. Por otra parte, esta evaluación por HIT califica de igual manera a un algoritmo que no alcanza un mejor valor que el máximo permitido.

Las parsimonias aplicadas durante el proceso evolutivo para los experimentos relacionados al PVV no afectan los resultados. Las parsimonias utilizadas en estos experimentos fueron para forzar o dirigir los algoritmos de acuerdo al conocimiento del problema para obtener soluciones factibles de éstos y para que el tamaño de los algoritmos no supere un número determinado. En todos los resultados obtenidos por los mejores algoritmos, las parsimonias tienen valor 0, esto permite dar pie a considerar incluir otras que posean características del problema.

La relación entre los experimentos por cada problema (1 con 2 y 3 con 4), no logra obtener una diferencia significativa en la calidad de los resultados, por lo que no es posible determinar que las muestras sean estadísticamente diferentes. Es decir, no es posible determinar que un método sea mejor que el otro. Sin embargo, una de las diferencias se encuentra en el tiempo que le toma al proceso evolutivo llevar a cabo cada una de las ejecuciones, donde el tiempo más significativo puede ser encontrado en los relacionados al PVV, éstos promedian 21 horas para el experimento 3 y 16 horas para el experimento 4. Con estos resultados surgen algunas preguntas que pueden ser planteadas para la PG utilizando co-evolución con islas que ameritarían una nueva experimentación, tales como: ¿cuál es el número óptimo de islas?, ¿cuál es el tamaño óptimo de la población global?, ¿deberían las islas tener la misma población?, ¿las funciones de evaluación utilizadas permiten seleccionar siempre a los mejores individuos o deben ser modificadas?, ¿es conveniente utilizar el método con co-evolución por la diferencia en el tiempo empleado en el proceso evolutivo?. Adicionalmente, al producir algoritmos que se generan de forma automática y presentan novedad, se abre la posibilidad de mejorar los elementos con el que fueron diseñados los experimentos, de esta forma surgen otras preguntas como: ¿el tamaño de las instancias utilizadas en el proceso evolutivo debe ser mayor?, ¿son el conjunto de funciones y terminales adecuados para resolver los problemas?, ¿existen elementos que considerar como parsimonia dentro de la función de evaluación?.
