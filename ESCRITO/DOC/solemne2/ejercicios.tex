Obtener la secuencia de rotaciones resultante de la inserción del conjunto de elementos

RESULTADO

    Se inserta el número 1 como raíz del árbol.
    Se inserta el número 2 a la derecha de la raíz.
    Se inserta el número 3 a la derecha del 2:Rotación simple a la izquierda sobre la raíz.
    Se inserta el número 4 a la derecha del 3.
    Se inserta el número 5 a la derecha del 4:Rotación simple a la izquierda sobre el 3.
    Se inserta el número 6 a la derecha del 5:Rotación simple a la izquierda sobre el 2 quedando el 4 como nodo raíz.
    Se inserta el número 7 a la derecha del 6:Rotación simple a la izquierda sobre el 5.
    Se inserta el número 15 a la derecha del 7.
    Se inserta el número 14 a la izquierda del 15:Doble rotación derecha-izquierda sobre 7.
    Se inserta el número 13 a la derecha del 7:Doble rotación derecha-izquierda sobre el 6.
    Se inserta el número 12 a la izquierda del 13:Rotación simple a la izquierda sobre el 4 quedando el 7 como nodo raíz.
    Se inserta el número 11 a la izquierda del 12:Rotación simple a la derecha sobre el 13.
    Se inserta el número 10 a la izquierda del 11:Rotación simple a la derecha sobre el 14.
    Se inserta el número 9 a la izquierda del 10:Rotación simple a la derecha sobre el 11.
    Se inserta el número 8 a la izquierda del 9.
