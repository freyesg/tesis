\resumenCastellano{

%Resumen 300 palabras máx.
En la actualidad a pesar de los múltiples estudios encontrados en la literatura, siguen existiendo problemas que no han podido ser resueltos de forma óptima en un tiempo polinomial. Entre los más conocidos y estudiados, se encuentran el problema del vendedor viajero y el problema de la mochila binaria. Estos problemas pertenecen al conjunto de problemas NP-Completo, lo que significa que no se ha determinado un algoritmo capaz de encontrar la solución óptima para cualquier instancia del problema con bajo tiempo computacional. Entre los métodos utilizados se encuentran los relacionados al proceso evolutivo, entre ellos la Programación Genética, la cual es utilizada para la generación automática de algoritmos mediante la combinación de distintos componentes elementales de heurísticas presentes en la literatura, utilizando conceptos básicos de la evolución como cruzamiento, mutación, reproducción, entre otros. Un concepto también proveniente de la evolución es la co-evolución, la cual ha comenzado a aparecer como complemento a los métodos que utilizan evolución en los últimos años. En este trabajo se evalúa el rendimiento computacional que poseen los algoritmos obtenidos mediante la Programación Genética tradicional en comparación a los algoritmos que se obtienen con la Programación Genética utilizando co-evolución. Para realizar la evaluación computacional se ha seleccionado un conjunto de instancias provenientes de la literatura. Luego de ejecutar los experimentos, se obtienen nuevas hiper-heurísticas para cada uno de los problemas mencionados que son evaluadas en un conjunto de instancias de evaluación y analizadas para comparar el rendimiento de éstas. Los mejores algoritmos para el problema de la mochila obtienen errores menores al 1\%, mientras que los algoritmos para el problema del vendedor viajero tienen un error aproximado de 7\%. Los resultados son similares en ambos métodos utilizados para cada uno de los problemas.


\vfill
\KeywordsES{
	%keywords 
	Programación Genética, co-evolución, problema del vendedor viajero, problema de la mochila.
	}
}

%\newpage

\resumenIngles{

Nowadays despite multiple studies found in the literature, problems remain that could not be solved optimally in polynomial time. Some of the best known and studied, are the traveling salesman problem and the problem of binary backpack. These problems belong to the set of NP-complete problems, meaning it has not been determined an algorithm able to find the optimal solution for any instance of the problem with low computational time. The methods used are those related to the evolutionary process, including the Genetic Programming, which is used for automatic generation of algorithms by combining different heuristics present in the literature using basic concepts of evolution as crossover, mutation, reproduction, among others. Another concept from evolution is the co-evolution, which has begun to appear as part of the methods that use evolution in recent years. In this document, the computational performance of algorithms obtained through traditional Genetic Programming compared to those obtained with Genetic Programming using coevolution is evaluated. To perform computational evaluation a set of instances has been selected from the literature. After running the experiments, the new hyper-heuristics obtained for each of the aforementioned problems are evaluated in a set of instances of evaluation and are analyzed to compare the performance of these. The best algorithms for the binary knapsack problem have errors less than 1\% while the algorithms for the traveling salesman problem have an approximate error of 7\%. The results are similar for both methods used for each of the problems.

% \vspace*{0.5cm}
\vfill
\KeywordsEN{
	Genetic Programming, coevolution, traveling salesman problem, binary knapsack problem.
	}
}


