%\section{Actualización de los pesos}
%\begin{frame}{\titulo}{\subtitulo}
%    \begin{figure}[H]
%        \centering
%        \scalebox{0.55}{%!tikz editor 1.0
\documentclass{article}
\usepackage{tikz}
\usepackage[graphics, active, tightpage]{preview}
\PreviewEnvironment{tikzpicture}

%!tikz preamble begin
\usetikzlibrary{shapes,snakes}

%!tikz preamble end


\begin{document}
%!tikz source begin
\begin{tikzpicture}
\tikzstyle{nodo}=[circle, draw, minimum size=1.25cm]
\tikzstyle{upw}=[dashed, red, ->, line width = 1pt]

\coordinate (l_0) at (0, 4.5);
\coordinate (l_1) at (7, 4.5);
\coordinate (l_2) at (14, 4.5);

\coordinate (f_1_1) at (0, 3); % CAPA ENTRADA
\coordinate (f_1_2) at (0, 0); % CAPA ENTRADA
\coordinate (f_1_3) at (0, -3); % CAPA ENTRADA
\coordinate (f_2_1) at (7, 2.5); % CAPA OCULTA 1
\coordinate (f_2_2) at (7, -2.5); % CAPA OCULTA 1
\coordinate (f_3_1) at (14, 0); % CAPA SALIDA

\node[] (l_0) at (l_0) {$L - 1$};
\node[] (l_1) at (l_1) {$L$};
\node[] (l_2) at (l_2) {$L + 1$};

\node[nodo] (f_1_1) at (f_1_1) {}; % CAPA ENTRADA
\node[nodo] (f_1_2) at (f_1_2) {}; % CAPA ENTRADA
\node[nodo] (f_1_3) at (f_1_3) {\small $f_i(e_i)$}; % CAPA ENTRADA
\node[nodo] (f_2_1) at (f_2_1) {\small $f_j(e_j)$}; % CAPA OCULTA 1
\node[nodo] (f_2_2) at (f_2_2) {}; % CAPA OCULTA 1
\node[nodo] (f_3_1) at (f_3_1) {\tiny $f_k(e_k)$}; % CAPA SALIDA


\draw[->, font=\scriptsize] (f_1_1) node[right of=f_1_1] {$a_{i}$} -- node[pos=0.75, above] (w_1_1) {$w^{l}_{ij}$} (f_2_1);
\draw[->, font=\scriptsize] (f_1_2) node[below right of=f_1_1] {$a_{i}$} -- node[pos=0.75, above] (w_1_2) {$w^{l}_{}$} (f_2_1);
\draw[->, font=\scriptsize] (f_1_3) -- node[pos=0.75, above] (w_1_3) {\scriptsize$w^{l}_{}$} (f_2_1);
\draw[->, font=\scriptsize] (f_1_1) -- node[pos=0.75, below] (w_2_1) {$w^{l}_{}$} (f_2_2);
\draw[->, font=\scriptsize] (f_1_2) -- node[pos=0.75, below] (w_2_2) {$w^{l}_{}$} (f_2_2);
\draw[->, font=\scriptsize] (f_1_3) -- node[pos=0.75, below] (w_2_3) {$w^{l}_{}$} (f_2_2);

\node[above of=f_2_1] {$e_{j} = \sum w_{ij}a_i$};

\draw[->, font=\scriptsize] (f_2_1) node[right of=f_2_1] {$a_{j}$} -- node[pos=0.75, above right] (w_j_k) {$w^{l+1}_{jk}$} (f_3_1);
\draw[->, font=\scriptsize] (f_2_2) -- node[pos=0.75, below right] {$w^{l+1}$} (f_3_1);
 
 
%\draw[upw] (f_3_1) to[bend left=20] (w_1_1);
%\draw[upw] (f_3_1) to[bend left=20] (w_1_2);
%\draw[upw] (f_3_1) to[bend left=20] node[pos=0.7, right] {$w^{l}_{ij} - \alpha\frac{\partial e}{\partial w}$} (w_1_3);


%\draw[upw] (f_4_1) to[bend right=20] node[pos=0.8, right] {$w^{l+1}_{jk} - \alpha\frac{\partial e}{\partial w^{l+1}_{jk}}$} (w_j_k);

%\node[above right of=w_j_k, node distance=3cm] {$\frac{\partial e}{\partial w^{l+1}_{jk}}=-(s_i - y_i)\frac{\partial y_i}{\partial w^{l+1}_{jk}}$};

\end{tikzpicture}
%!tikz source end

\end{document}}
%    \end{figure}
%
%    \begin{itemize}
%        \item $W^{l}_{ij}$ es el peso de la conexión entre la neurona $i$ de la capa $l$ con la neurona $j$ de la capa $l + 1$.
%    \end{itemize}
%\end{frame}
%
%\section{El algoritmo de retropropagación}
%\begin{frame}{\titulo}{\subtitulo}
%    \begin{itemize}
%        \item Para cualquier problema de aprendizaje supervisado queremos encontrar un conjunto de pesos $W$ que minimice la salida de $J(W)$.\bigskip
%
%        \item Cada neurona es una función de la neurona anterior conectada a ella.
%
%        \item Si uno cambiara el valor de $w_1$, las neuronas "ocultas 1" y "ocultas 2" cambiarían.
%
%        \item Debido a esta noción de dependencias funcionales, podemos formular matemáticamente el resultado como una función compuesta extensa:
%        \begin{eqnarray*}
%            output &=&  act(w_3*h_2)\\
%            h_2 &=&     act(w_2*h_1)\\
%            h_1 &=&     act(w_1*input)\\
%            &\Downarrow&\\
%            output &=&  act(w_3*act(w_2*act(w_1*input)))
%        \end{eqnarray*}
%
%        \item La salida es una función compuesta de los pesos, entradas y funciónes de activación.
%    \end{itemize}
%\end{frame}

%\begin{frame}{\titulo}{\subtitulo}
%\begin{itemize}
%    \item Si tomáramos entonces la derivada de dicha función con respecto a algún peso arbitrario, aplicaríamos iterativamente la regla de la cadena. El resultado sería similar al siguiente
%    \begin{eqnarray*}
%    \frac{\partial output}{\partial w_1}
%    =
%    \frac{\partial output}{\partial h_2}
%    \frac{\partial h_2}{\partial h_1}
%    \frac{\partial h_1}{\partial w_1}
%    \end{eqnarray*}
%
%    \item Ahora, vamos a adjuntar una caja negra a la cola de nuestra red neuronal. Esta caja negra calculará y devolverá el error - usando la función de coste - de nuestra salida
%    \begin{figure}[H]
%        \centering
%        \scalebox{0.55}{% http://home.agh.edu.pl/~vlsi/AI/backp_t_en/backprop.html
% http://home.agh.edu.pl/~vlsi/AI/backp_t_en/backprop.html
%\ifx\du\undefined
%  \newlength{\du}
%\fi
%\newcommand{\nweigth}[2]{$W^{#1}_{#2} - \alpha\frac{\partial J}{\partial W^{#1}_{#2}}$}
%\newcommand{\w}[2]{$W^{#1}_{#2}$}

%\setlength{\du}{15\unitlength}
\begin{tikzpicture}
\tikzstyle{neuron}=[circle,draw, minimum size=4em, font=\Large]
\tikzstyle{update}=[dashed, blue]

\coordinate (x_1) at (-5, 1.5);
\coordinate (x_2) at (-5, -1.5);

\coordinate (n_1_1) at (-1, 3);
\coordinate (n_1_2) at (-1, 0);
\coordinate (n_1_3) at (-1, -3);

\coordinate (n_2_1) at (5, 1.5);
\coordinate (n_2_2) at (5, -1.5);

\coordinate (n_3_1) at (11, 0);


\node[neuron] (x_1) at (x_1) {$x_{1}$};
\node[neuron] (x_2) at (x_2) {$x_{2}$};

\node[neuron] (n_1_1) at (n_1_1) {$f^{1}_{1}(e)$};
\node[neuron] (n_1_2) at (n_1_2) {$f^{1}_{2}(e)$};
\node[neuron] (n_1_3) at (n_1_3) {$f^{1}_{3}(e)$};

\node[neuron] (n_2_1) at (n_2_1) {$f^{2}_{1}(e)$};
\node[neuron] (n_2_2) at (n_2_2) {$f^{2}_{2}(e)$};

\node[neuron] (n_3_1) at (n_3_1) {$f^{3}_{1}(e)$};


\node[above of=n_1_1] {\Large $y_i = f^{1}_{k}(\sum w_i x_i)$};
\node[above of=n_2_1] {\Large $y_i = f^{1}_{k}(\sum w_i x_i)$};
%\node[above of=n_3_1] {\Large $y_i = f^{1}_{k}(\sum w_i x_i)$};

\draw[->] (x_1) -- (n_1_1);
\draw[->] (x_1) -- (n_1_2);
\draw[->] (x_1) -- (n_1_3);
\draw[->] (x_2) -- (n_1_1);
\draw[->] (x_2) -- (n_1_2);
\draw[->] (x_2) -- (n_1_3);

\draw[->] (n_1_1) -- (n_2_1);
\draw[->] (n_1_1) -- (n_2_2);
\draw[->] (n_1_2) -- (n_2_1);
\draw[->] (n_1_2) -- (n_2_2);
\draw[->] (n_1_3) -- (n_2_1);
\draw[->] (n_1_3) -- (n_2_2);

\draw[->] (n_2_1) -- (n_3_1);
\draw[->] (n_2_2) -- (n_3_1);

\end{tikzpicture}
}
%    \end{figure}
%\end{itemize}
%\end{frame}


%\begin{frame}
%    \begin{itemize}
%        \item Cada uno de estos derivados puede simplificarse una vez que elegimos una función de activación y error, de tal manera que el resultado completo representaría un valor numérico.\bigskip
%
%        \item En ese punto, cualquier abstracción se ha eliminado, y la derivada de error se puede utilizar en el descenso del gradiente para mejorar iterativamente el peso.\bigskip
%
%        \item Se calculan las derivadas de error cada otro peso en la red y aplicar descenso gradiente de la misma manera.\bigskip
%
%        \item Esto es retropropagación - simplemente el cálculo de derivados que se alimentan a un algoritmo de optimización convexa.\bigskip
%
%        \item Lo llamamos "retropropagación" porque casi parece como si estuviéramos atravesando desde el error de salida a los pesos, tomando pasos iterativos usando la cadena de la regla hasta que "alcancemos" nuestro peso.
%    \end{itemize}
%\end{frame}

\section{El Gradiente Descendente}
\begin{frame}{\titulo}{\subtitulo}
    \begin{itemize}
        \item Se busca un algoritmo que permita encontrar pesos y sesgos para que la salida de la red aproxime los valores de $y(x)$ a los valores correspondientes con cada entrada $x$.\bigskip

        \item Para cuantificar qué tan bien estamos logrando este objetivo definimos una función de costo

    \end{itemize}
\end{frame}


\section{El Gradiente Descendente Estocástico}
\begin{frame}{\titulo}{\subtitulo}
    \begin{itemize}
        \item El método del gradiente descendente estocástico (SGD) actualiza los parámetros en cada ejemplo $x_i$ y etiqueta $y_i$ de la siguiente manera $$ \theta = \theta - \eta\nabla_\theta $$
    \end{itemize}
\end{frame}



\begin{frame}{\titulo}{\subtitulo}
    \begin{figure}[H]
        \centering
        \scalebox{0.6}{% http://home.agh.edu.pl/~vlsi/AI/backp_t_en/backprop.html
% http://home.agh.edu.pl/~vlsi/AI/backp_t_en/backprop.html
%\ifx\du\undefined
%  \newlength{\du}
%\fi
%\newcommand{\nweigth}[2]{$W^{#1}_{#2} - \alpha\frac{\partial J}{\partial W^{#1}_{#2}}$}
%\newcommand{\w}[2]{$W^{#1}_{#2}$}

%\setlength{\du}{15\unitlength}
\begin{tikzpicture}
\tikzstyle{neuron}=[circle,draw, minimum size=4em, font=\Large]
\tikzstyle{update}=[dashed, blue]

\coordinate (x_1) at (-5, 1.5);
\coordinate (x_2) at (-5, -1.5);

\coordinate (n_1_1) at (-1, 3);
\coordinate (n_1_2) at (-1, 0);
\coordinate (n_1_3) at (-1, -3);

\coordinate (n_2_1) at (5, 1.5);
\coordinate (n_2_2) at (5, -1.5);

\coordinate (n_3_1) at (11, 0);


\node[neuron] (x_1) at (x_1) {$x_{1}$};
\node[neuron] (x_2) at (x_2) {$x_{2}$};

\node[neuron] (n_1_1) at (n_1_1) {$f^{1}_{1}(e)$};
\node[neuron] (n_1_2) at (n_1_2) {$f^{1}_{2}(e)$};
\node[neuron] (n_1_3) at (n_1_3) {$f^{1}_{3}(e)$};

\node[neuron] (n_2_1) at (n_2_1) {$f^{2}_{1}(e)$};
\node[neuron] (n_2_2) at (n_2_2) {$f^{2}_{2}(e)$};

\node[neuron] (n_3_1) at (n_3_1) {$f^{3}_{1}(e)$};


\node[above of=n_1_1] {\Large $y_i = f^{1}_{k}(\sum w_i x_i)$};
\node[above of=n_2_1] {\Large $y_i = f^{1}_{k}(\sum w_i x_i)$};
%\node[above of=n_3_1] {\Large $y_i = f^{1}_{k}(\sum w_i x_i)$};

\draw[->] (x_1) -- (n_1_1);
\draw[->] (x_1) -- (n_1_2);
\draw[->] (x_1) -- (n_1_3);
\draw[->] (x_2) -- (n_1_1);
\draw[->] (x_2) -- (n_1_2);
\draw[->] (x_2) -- (n_1_3);

\draw[->] (n_1_1) -- (n_2_1);
\draw[->] (n_1_1) -- (n_2_2);
\draw[->] (n_1_2) -- (n_2_1);
\draw[->] (n_1_2) -- (n_2_2);
\draw[->] (n_1_3) -- (n_2_1);
\draw[->] (n_1_3) -- (n_2_2);

\draw[->] (n_2_1) -- (n_3_1);
\draw[->] (n_2_2) -- (n_3_1);

\end{tikzpicture}
}
    \end{figure}
\end{frame}

\subsection{El desvanecimiento del gradiente}
\begin{frame}{\titulo}{\subtitulo}
	\begin{figure}[H]
        \centering
        %\scalebox{0.6}{% http://staff.itee.uq.edu.au/janetw/cmc/chapters/BackProp/index2.html
% http://outlace.com/Beginner-Tutorial-Backpropagation/
% http://neuralnetworksanddeeplearning.com/chap2.html
% http://staff.itee.uq.edu.au/janetw/cmc/chapters/BackProp/slides/Backprop_files/frame.htm
% http://staff.itee.uq.edu.au/janetw/cmc/chapters/BackProp/
% http://home.agh.edu.pl/~vlsi/AI/backp_t_en/backprop.html
% https://www.google.cl/webhp?sourceid=chrome-instant&ion=1&espv=2&ie=UTF-8#safe=off&q=backpropagation+practice+example
% https://ayearofai.com/rohan-4-the-vanishing-gradient-problem-ec68f76ffb9b#.9ntv81akz

\ifx\du\undefined
  \newlength{\du}
\fi
\newcommand{\nweigth}[2]{$W^{#1}_{#2} - \alpha\frac{\partial J}{\partial W^{#1}_{#2}}$}
\newcommand{\w}[2]{$W^{#1}_{#2}$}

\setlength{\du}{1\unitlength}
\begin{tikzpicture}[font=\small]
\tikzstyle{neuron}=[circle,draw, minimum size=2em]
\tikzstyle{update}=[dashed, blue]

\pgftransformxscale{1.000000}
\pgftransformyscale{-1.000000}

\coordinate (x1)    at (0.000000\du, 0.000000\du);
\coordinate (x2)    at (0.000000\du, 8.000000\du);

\coordinate (i)     at (3.000000\du, 0.000000\du);
\coordinate (ii)    at (3.000000\du, 8.000000\du);

\coordinate (iv)    at (10.000000\du, 0.000000\du);
\coordinate (v)     at (10.000000\du, 8.000000\du);

\coordinate (vii)   at (17.000000\du, 0.000000\du);
\coordinate (viii)  at (17.000000\du, 8.000000\du);

\coordinate (ix)   at (17.000000\du, 0.000000\du);
\coordinate (diez)  at (17.000000\du, 8.000000\du);


\coordinate (x)     at (24.000000\du, 4.000000\du);
\coordinate (y)     at (27.000000\du, 4.000000\du);


%%%%%%%%%%%%%%%%%%%%%%%%%%%%%%%%%%%%%%%%%%%%%%%%%%%%%%%%%%
% ENTRADA
\node (X1) at (x1) {\Huge $x_1$};
\node (X2) at (x2) {\Huge $x_2$};

% CAPA DE ENTRADA
\node[neuron] (A) at  (i) {}; % A
\node[neuron] (B) at  (ii)  {}; % B

% CAPA 1
\node[neuron] (C) at  (iv)  {}; % C
\node[neuron] (D) at  (v)  {}; % D

% CAPA 2
\node[neuron] (E) at  (vii)  {}; % E
\node[neuron] (F) at  (viii)  {}; % F

% CAPA 2
\node[neuron] (H) at  (ix)  {}; % E
\node[neuron] (I) at  (diez)  {}; % F

% CAPA DE SALIDA
\node[neuron] (G) at  (x)  {}; % G

% SALIDA
\node (out) at (y) {\Huge $y_1$};
%%%%%%%%%%%%%%%%%%%%%%%%%%%%%%%%%%%%%%%%%%%%%%%%%%%%%%%%%%


\draw[->] (X1) -- (A);
\draw[->] (X2) -- (B);


%%%%%%%%%%%%%%%%%%%%%%%%%%%%%%%%%%%%%%%%%%%%%%%%%%%%%%%%%%
%%%%%%%%%%%%%%%%%%%%%%%%%%%%%%%%%%%%%%%%%%%%%%%%%%%%%%%%%%
\draw[->] (A) -- node[above, pos=0.2] (W_1_11) {\w{1}{11}} (C);
\draw[->] (A) -- node[left , pos=0.2] (W_1_12) {\w{1}{12}} (D);
\draw[->] (B) -- node[left , pos=0.2] (W_1_13) {\w{1}{13}} (C);
\draw[->] (B) -- node[below, pos=0.2] (W_1_14) {\w{1}{14}} (D);

%\draw[->, dashed, blue] (C) to [bend left=40] node[above] {$W^{1}_{11} - \alpha\frac{\partial J}{\partial W^{1}_{11}}$} (W_1_11);
\draw[->, update] (C) to [bend left=40] node[above] {\nweigth{1}{11}} (W_1_11);
\draw[->, update] (D) to [bend left=50] node[above right, pos=0.98] {\nweigth{1}{12}} (W_1_12);

\draw[->, update] (C) to [bend right=50] node[below right, pos=0.98] {\nweigth{1}{13}} (W_1_13);
\draw[->, update] (D) to [bend right=40] node[below] {\nweigth{1}{14}} (W_1_14);

%%%%%%%%%%%%%%%%%%%%%%%%%%%%%%%%%%%%%%%%%%%%%%%%%%%%%%%%%%
%%%%%%%%%%%%%%%%%%%%%%%%%%%%%%%%%%%%%%%%%%%%%%%%%%%%%%%%%%
\draw[->] (C) -- node[above, pos=0.2] (W_2_11) {\w{l}{11}} (E);
\draw[->] (C) -- node[left , pos=0.2] (W_2_12) {\w{2}{12}} (F);
\draw[->] (D) -- node[left , pos=0.2] (W_2_13) {\w{2}{13}} (E);
\draw[->] (D) -- node[below, pos=0.2] (W_2_14) {\w{2}{14}} (F);

\draw[->, update] (E) to [bend left=40] node[above] {\nweigth{2}{11}} (W_2_11);
\draw[->, update] (F) to [bend left=50] node[above right, pos=0.98] {\nweigth{2}{12}} (W_2_12);

\draw[->, update] (E) to [bend right=50] node[below right, pos=0.98] {\nweigth{2}{13}} (W_2_13);
\draw[->, update] (F) to [bend right=40] node[below] {\nweigth{2}{14}} (W_2_14);

%%%%%%%%%%%%%%%%%%%%%%%%%%%%%%%%%%%%%%%%%%%%%%%%%%%%%%%%%%
%%%%%%%%%%%%%%%%%%%%%%%%%%%%%%%%%%%%%%%%%%%%%%%%%%%%%%%%%%
\draw[->] (E) -- node[right, pos=0.2] (W_3_11) {\w{3}{11}} (G);
\draw[->] (F) -- node[right, pos=0.2] (W_3_12) {\w{3}{12}} (G);

\draw[->, update] (G) to [bend  left=50] node[above right, pos=0.98] {\nweigth{3}{11}} (W_3_11);
\draw[->, update] (G) to [bend right=40] node[below right, pos=0.98] {\nweigth{3}{12}} (W_3_12);


\draw[->] (G) -- (out);
\end{tikzpicture}
}
        \scalebox{0.50}{\begin{tikzpicture}

	\tikzstyle{nodo}=[circle, draw, minimum size=1.25cm]
	\tikzstyle{upw}=[dashed, red, ->, line width = 1pt]

	\coordinate (l_0) at (0, 5.0);
	\coordinate (l_1) at (5.5, 5.0);
	\coordinate (l_2) at (10, 5.0);

	\coordinate (f_1_1) at (0, 3.5); % CAPA ENTRADA
	\coordinate (f_1_2) at (0, 0); % CAPA ENTRADA
	\coordinate (f_1_3) at (0, -3.5); % CAPA ENTRADA
	\coordinate (f_2_1) at (6.0, 3.5); % CAPA OCULTA 1
	\coordinate (f_2_2) at (6.0, -3.5); % CAPA OCULTA 1
	\coordinate (f_3_1) at (12, 0); % CAPA SALIDA
	\coordinate (y) at (14, 0); % CAPA SALIDA

	\node[] (l_0) at (l_0) {$L - 1$};
	\node[] (l_1) at (l_1) {$L$};
	\node[] (l_2) at (l_2) {$L + 1$};

	\node[nodo] (f_1_1) at (f_1_1) {}; % CAPA ENTRADA
	\node[nodo] (f_1_2) at (f_1_2) {}; % CAPA ENTRADA
	\node[nodo] (f_1_3) at (f_1_3) {\small $f_i(e_i)$}; % CAPA ENTRADA
	\node[nodo] (f_2_1) at (f_2_1) {\small $f_j(e_j)$}; % CAPA OCULTA 1
	\node[nodo] (f_2_2) at (f_2_2) {}; % CAPA OCULTA 1
	\node[nodo] (f_3_1) at (f_3_1) {\tiny $f_k(e_k)$}; % CAPA SALIDA
	\node[] (y) at (y) {$y$}; % CAPA SALIDA

	\node[left of=f_1_1, node distance=1.4cm] {\LARGE $\cdots$};
	\node[left of=f_1_2, node distance=1.4cm] {\LARGE $\cdots$};
	\node[left of=f_1_3, node distance=1.4cm] {\LARGE $\cdots$};

	\draw[->] (f_1_1) -- node[pos=0.6, above] (w_1_1) {$w^{l}_{ij}$} (f_2_1);
	\draw[->] (f_1_2) -- node[pos=0.6, above] (w_1_2) {$w^{l}_{}$} (f_2_1);
	\draw[->] (f_1_3) -- node[pos=0.6, above] (w_1_3) {$w^{l}_{}$} (f_2_1);
	\draw[->] (f_1_1) node[above right of=f_1_2] {$a_{i}$} -- node[pos=0.6, below] (w_2_1) {$w^{l}_{}$} (f_2_2);
	\draw[->] (f_1_2) node[below right of=f_1_2] {$a_{i}$} -- node[pos=0.6, below] (w_2_2) {$w^{l}_{}$} (f_2_2);
	\draw[->] (f_1_3) -- node[pos=0.6, below] (w_2_3) {$w^{l}_{}$} (f_2_2);
	\draw[->] (f_2_1) node[right of=f_2_1] {$a_{j}$} -- node[pos=0.5, above right] (w_j_k) {$w^{l+1}_{jk}$} (f_3_1);
	\draw[->] (f_2_2) -- node[pos=0.5, below right] {$w^{l+1}$} (f_3_1);
	\draw[->] (f_3_1) -- node[pos=0.5, below right] {$$} (y);

	\node[above right of=f_2_1, node distance=1.5cm] {$e_{j} = \sum w_{ij}a_i$};


	\draw[upw] (f_3_1) to[bend right=50] node[above right, pos=0.8] {$w^{l+1}_{ik} - \alpha e^{l+1}_{i}x^{l}_{k}$} (w_j_k);
	\draw[upw] (f_2_1) to[bend right=30] node[above]{$$} (w_1_1);
	\draw[upw] (f_2_1) to[bend right=30] node[above]{$$} (w_1_2);
	\draw[upw] (f_2_1) to[bend right=30] node[above]{$$} (w_1_3);


	\node[blue, above right of=f_3_1] (e_i) {$\delta^{l+1}_{k}$};
	\node[blue, below of=f_2_1] (e_2_1_i) {$\delta^{l}_{j}$};
	\node[blue, above of=f_2_2] (e_2_2_i) {$\delta^{l}$};
	\node[blue, right of=f_1_2] (e_1_2_i) {$\delta^{l-1}$};

	\draw[upw, blue] (y) to[bend right=40] node[right]{$f'(y)$} (e_i);
	\draw[upw, blue] (f_3_1) to[bend left=10] node[below left, pos=0.7]{$f'(a_{j})\sum_{i}w^{l}_{ji}\delta^{l+1}_{i}$} (e_2_1_i);
	\draw[upw, blue] (f_3_1) to[bend right=10] (e_2_2_i);

	\draw[upw, blue] (f_2_1) to[bend left=5] (e_1_2_i);
	\draw[upw, blue] (f_2_2) to[bend right=5] (e_1_2_i);

\end{tikzpicture}
}
    \end{figure}
	%$$ f'(a_j) \sum_{i} w_{ji}^{l} \delta_{i}^{l + 1} $$
	$$ f'(a_j) \sum_{i} w_{ji}^{l} f'(y) $$
	$$ \underbrace{f_{a_{j}}'(a_j) \sum_{m} w_{jm}^{l} \overbrace{\left(f'(a_j) \sum_{n} w_{jn}^{l} f'(y)\right)}^{\delta_{k}^{l + 1}}}_{\delta_{j}^{l}} $$
\end{frame}
