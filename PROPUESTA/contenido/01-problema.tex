\section{Descripción del problema}
La autorregulación del flujo sanguíneo muestra la capacidad de un órgano para mantener una perfusión\footnote{Introducción lenta y continuada de una sustancia medicamentosa o de sangre en un organismo u órgano por vía intravenosa, subcutánea o rectal.} constante frente a los cambios de presión arterial. El caso particular de la autorregulación cerebral ({\em Cerebral Autorregulation}, \ca), se puede definir en términos de los cambios en la resistencia vascular, y su alteración ha atraído una atención especial en el campo cerebrovascular. La \ca\, opera con una presión arterial media ({\em Media Arterial Blood Pressure}, \pam) del orden de los 60 y 150 mmHg. Estos límites no son totalmente fijos y puede ser modulada por la actividad nervioso-simpático, o cualquier factor que disminuya o aumente el flujo sanguíneo cerebral ({\em Cerebral Blood Flow}, \cbf), pero en particular los cambios en la presión arterial de dióxido de carbono. Dentro de estos límites, la \ca\, protege al cerebro de la isquemia debido a la hipotensión y también previene el daño capilar, debido a los aumentos repentinos de la presión arterial.

Los primeros estudios de autorregulación cerebral en seres humanos se basaron en métodos que permitían la medición del \cbf\,, pero no proporcionaban información sobre el tiempo que tardaba en recuperarse el \cbf\, luego de un cambio brusco en la presión sanguínea ({\em Blood Pressure}, \bp) \citep{Lassen1959}. Esta limitación fue superada con la introducción de la {\em <<Ecografía Doppler Transcranial>>} ({\em TCD}), la que permitía visualizar la velocidad del flujo sanguíneo cerebral ({\em Cerebral blood flow velocity}, \cbfv) frente a los cambios repentinos de la \pam. Este nuevo enfoque para estudiar la autorregulación cerebral se denominó autorregulación cerebral dinámica ({\em Dynamic cerebral autorregulation}, \dca), en contraste con los métodos basados en mediciones del \cbf\, y \bp\, promedio durante varios minutos que ahora se denominan autorregulación cerebral estática \citep{Aaslid1989, Panerai1998a, Tiecks1995a}.

El análisis de la función de transferencia de los cambios en la \bp\ y en el \cbf\, permiten medir la relación dinámica entre ambos. Las amplitudes, que se describen por dicha oscilación, son pequeñas, varian en forma considerable y no pueden ser controlados, por eso su importancia clínica ha sido cuestionada.

El análisis espectral de una señal permite cuantificar las oscilaciones de la \bp\, y de la \cbfv\, y estimar la fase, la función de transferencia, la amplitud (ganancia) y la coherencia. Se ha sugerido que la autorregulación cerebral es más eficaz en frecuencias bajas en comparación con altas frecuencias \citep{Diehl1995, Diehl1998, Giller1990, Panerai1998b, Zhang1998}, sin embargo, el método tiene como limitante lo pequeño que resultan las magnitudes de las oscilaciones de la \bp\, y la \cbfv.

\subsection{Motivación}
En los estudios de la autorregulación con modelos lineales, los parámetros utilizados se obtienen desde la frecuencia, el impulso, o la etapa de respuesta. Algunos de los parámetros utilizados incluyen la ganancia y la fase \citep{Zhang1998, Birch1995, Panerai1999, Liu2003}. Un método alternativo para medir la autorregulación es el índice ARI \citep{Tiecks1995a}, que varía desde 0 (ausencia de autorregulación) y 9 (buena autorregulación). Por otra parte, la función de transferencia en el análisis líneal y en modelos Volterra en el análisis no lineal, han demostrado que la autorregulación cerebral es más eficaz en el espectro de frecuencias por debajo de \hz{0.1}, donde se encuentra gran parte de la potencia del espectro de la \abp, es decir, cambios espontáneos en la \pam\, no provocan grandes variaciones sobre la \cbfv\, media (\mcbfv).

Dado que la cavidad cerebrovascular es controlada por mecanismos metabólicos y miogénicos\footnote{mecanismo por el cual las arterias y arteriolas reaccionan ante un aumento o descenso de la presión arterial para mantener el flujo sanguineo dentro de lo normal en los vasos sanguíneos}, relaciones con el endotelio y mecanismos neuronales \citep{Faraci1998, Panerai1999, Paulson1990} es que la dinámica de la autorregulación cerebral son ampliamente activos en diferentes bandas de frecuencia, desde \hz{0.005} hasta los \hz{0.5}. \cite{Zhang1998} muestra que los mecanismos metabólicos son más activos a frecuencias muy bajas; los mecanismos miogénicos son más activos a altas frecuencias, mientras que los mecanísmos relacionados del endotelio se encuentra entre las bandas de frecuencias intermedias.

La literatura describe técnicas que permiten estimar la autorregulación logrando representar la relación existente entre la \pam\, y la \cbfv. La mayoría de los estudios de autorregulación se enfocan en el uso de métodos lineales \citep{Zhang1998, Birch1995, Panerai1999, Simpson2001, Panerai2004} mientras que algunos trabajos como los de \cite{Panerai1999, Panerai2004, Mitsis2004a, Mitsis2002, Angarita2011} desarrollaron métodos no lineales. A pesar de que las técnicas no lineales proporcionan un modelo de ajuste mejor, no se ha comprobado su efectividad evaluando la autorregulación frente a las técnicas lineales y en distintos rangos de frecuencia.

\subsection{Definición del problema}
Para el estudio de la \ca\, dinámica se han utilizado diversas técnicas para inducir cambios rápidos en la \pam. Estos incluyen la deflación repentina de puños apretados \citep{Aaslid1989, Tiecks1995a}, las maniobras de Valsalva \citep{Greenfield1984, Tiecks1995a}, la respiración forzada \citep{Diehl1995}, en cuclillas periódica \citep{Birch1995}, o la inclinación \citep{Anthony1993}. Otros investigadores se han basado en las fluctuaciones espontáneas en \pam\, para observar los correspondientes cambios transitorios en la velocidad del \cbf.

Los actuales modelos de entrenamiento, que permiten evaluar el proceso de la autorregulación, realizan su trabajo utilizando todo el espectro de frecuencias que se puede encontrar durante el proceso. \cite{Zhang1998} muestra que los mecanismos metabólicos presentan mayor actividad en frecuencias muy bajas, que figuran entre los 0.07 y los 0.3Hz, mientras que los mecanísmos miogénicos son más activos a altas frecuencias, mientras que los mecanísmos relacionados del endotelio se encuentra entre las bandas de frecuencias intermedias. Entonces surge la duda sobre los resultados que se obtendrán si se filtran los rangos de frecuencias asociados sólo al proceso de autorregulación.

Se desconoce a priori cómo es que aprenden los distintos modelos frente a señales alteradas y en distintos espectros de frecuencias y si acaso serán más eficientes que los modelos entrenados en todo el espectro de frecuencias.
