\documentclass[informe]{tesis-usach}

\usepackage{tikz}
\usetikzlibrary{arrows, calc, decorations.markings, math}
\usetikzlibrary{matrix, chains, positioning, decorations.pathreplacing, shapes, snakes}
\usepackage{pgfplots}


\title{Desempeño de redes neuronales entrenadas mediante simulated annealing frente a otros métodos}
\informe{Informe Final\\Propuesta de tesis}
\facultad{Facultad de Ingeniería}
\departamento{Departamento de ingeniería informática}
\author{Felipe Alberto Reyes González}
\programa{Magíster en Ingeniería Informática}
\profesor{Victor Parada}
\celular{890 26 317}
\correo{felipe.reyesg@usach.cl}
\date{\today}

\usepackage{amsmath}
% TIPOS DE ENTRADAS : http://newton.ex.ac.uk/tex/pack/bibtex/btxdoc/node6.html
% CAMPOS : http://newton.ex.ac.uk/tex/pack/bibtex/btxdoc/node7.html
\begin{document}
\renewcommand{\contentsname}{Tabla de contenido}
\renewcommand{\refname}{Bibliografía}
\renewcommand{\appendixname}{Apéndice}
\renewcommand{\appendixtocname}{Apéndices}
\renewcommand{\appendixpagename}{Apéndices}
\renewcommand{\tablename}{Tabla}

% helvetica en el cuerpo del documento
% \sffamily

% \frontmatter
\maketitle
% \aditionalpages
% \indice
% \mainmatter

%%%%%%%%%%%%%%%%%%%%%%%%%%%%
% Inicio cosas importante
%%%%%%%%%%%%%%%%%%%%%%%%%%%%
%
\tableofcontents
\newpage

\section{Descripción del problema}
La autorregulación del flujo sanguíneo muestra la capacidad de un órgano para mantener una perfusión\footnote{Introducción lenta y continuada de una sustancia medicamentosa o de sangre en un organismo u órgano por vía intravenosa, subcutánea o rectal.} constante frente a los cambios de presión arterial. El caso particular de la autorregulación cerebral ({\em Cerebral Autorregulation}, \ca), se puede definir en términos de los cambios en la resistencia vascular, y su alteración ha atraído una atención especial en el campo cerebrovascular. La \ca\, opera con una presión arterial media ({\em Media Arterial Blood Pressure}, \pam) del orden de los 60 y 150 mmHg. Estos límites no son totalmente fijos y puede ser modulada por la actividad nervioso-simpático, o cualquier factor que disminuya o aumente el flujo sanguíneo cerebral ({\em Cerebral Blood Flow}, \cbf), pero en particular los cambios en la presión arterial de dióxido de carbono. Dentro de estos límites, la \ca\, protege al cerebro de la isquemia debido a la hipotensión y también previene el daño capilar, debido a los aumentos repentinos de la presión arterial.

Los primeros estudios de autorregulación cerebral en seres humanos se basaron en métodos que permitían la medición del \cbf\,, pero no proporcionaban información sobre el tiempo que tardaba en recuperarse el \cbf\, luego de un cambio brusco en la presión sanguínea ({\em Blood Pressure}, \bp) \citep{Lassen1959}. Esta limitación fue superada con la introducción de la {\em <<Ecografía Doppler Transcranial>>} ({\em TCD}), la que permitía visualizar la velocidad del flujo sanguíneo cerebral ({\em Cerebral blood flow velocity}, \cbfv) frente a los cambios repentinos de la \pam. Este nuevo enfoque para estudiar la autorregulación cerebral se denominó autorregulación cerebral dinámica ({\em Dynamic cerebral autorregulation}, \dca), en contraste con los métodos basados en mediciones del \cbf\, y \bp\, promedio durante varios minutos que ahora se denominan autorregulación cerebral estática \citep{Aaslid1989, Panerai1998a, Tiecks1995a}.

El análisis de la función de transferencia de los cambios en la \bp\ y en el \cbf\, permiten medir la relación dinámica entre ambos. Las amplitudes, que se describen por dicha oscilación, son pequeñas, varian en forma considerable y no pueden ser controlados, por eso su importancia clínica ha sido cuestionada.

El análisis espectral de una señal permite cuantificar las oscilaciones de la \bp\, y de la \cbfv\, y estimar la fase, la función de transferencia, la amplitud (ganancia) y la coherencia. Se ha sugerido que la autorregulación cerebral es más eficaz en frecuencias bajas en comparación con altas frecuencias \citep{Diehl1995, Diehl1998, Giller1990, Panerai1998b, Zhang1998}, sin embargo, el método tiene como limitante lo pequeño que resultan las magnitudes de las oscilaciones de la \bp\, y la \cbfv.

\subsection{Motivación}
En los estudios de la autorregulación con modelos lineales, los parámetros utilizados se obtienen desde la frecuencia, el impulso, o la etapa de respuesta. Algunos de los parámetros utilizados incluyen la ganancia y la fase \citep{Zhang1998, Birch1995, Panerai1999, Liu2003}. Un método alternativo para medir la autorregulación es el índice ARI \citep{Tiecks1995a}, que varía desde 0 (ausencia de autorregulación) y 9 (buena autorregulación). Por otra parte, la función de transferencia en el análisis líneal y en modelos Volterra en el análisis no lineal, han demostrado que la autorregulación cerebral es más eficaz en el espectro de frecuencias por debajo de \hz{0.1}, donde se encuentra gran parte de la potencia del espectro de la \abp, es decir, cambios espontáneos en la \pam\, no provocan grandes variaciones sobre la \cbfv\, media (\mcbfv).

Dado que la cavidad cerebrovascular es controlada por mecanismos metabólicos y miogénicos\footnote{mecanismo por el cual las arterias y arteriolas reaccionan ante un aumento o descenso de la presión arterial para mantener el flujo sanguineo dentro de lo normal en los vasos sanguíneos}, relaciones con el endotelio y mecanismos neuronales \citep{Faraci1998, Panerai1999, Paulson1990} es que la dinámica de la autorregulación cerebral son ampliamente activos en diferentes bandas de frecuencia, desde \hz{0.005} hasta los \hz{0.5}. \cite{Zhang1998} muestra que los mecanismos metabólicos son más activos a frecuencias muy bajas; los mecanismos miogénicos son más activos a altas frecuencias, mientras que los mecanísmos relacionados del endotelio se encuentra entre las bandas de frecuencias intermedias.

La literatura describe técnicas que permiten estimar la autorregulación logrando representar la relación existente entre la \pam\, y la \cbfv. La mayoría de los estudios de autorregulación se enfocan en el uso de métodos lineales \citep{Zhang1998, Birch1995, Panerai1999, Simpson2001, Panerai2004} mientras que algunos trabajos como los de \cite{Panerai1999, Panerai2004, Mitsis2004a, Mitsis2002, Angarita2011} desarrollaron métodos no lineales. A pesar de que las técnicas no lineales proporcionan un modelo de ajuste mejor, no se ha comprobado su efectividad evaluando la autorregulación frente a las técnicas lineales y en distintos rangos de frecuencia.

\subsection{Definición del problema}
Para el estudio de la \ca\, dinámica se han utilizado diversas técnicas para inducir cambios rápidos en la \pam. Estos incluyen la deflación repentina de puños apretados \citep{Aaslid1989, Tiecks1995a}, las maniobras de Valsalva \citep{Greenfield1984, Tiecks1995a}, la respiración forzada \citep{Diehl1995}, en cuclillas periódica \citep{Birch1995}, o la inclinación \citep{Anthony1993}. Otros investigadores se han basado en las fluctuaciones espontáneas en \pam\, para observar los correspondientes cambios transitorios en la velocidad del \cbf.

Los actuales modelos de entrenamiento, que permiten evaluar el proceso de la autorregulación, realizan su trabajo utilizando todo el espectro de frecuencias que se puede encontrar durante el proceso. \cite{Zhang1998} muestra que los mecanismos metabólicos presentan mayor actividad en frecuencias muy bajas, que figuran entre los 0.07 y los 0.3Hz, mientras que los mecanísmos miogénicos son más activos a altas frecuencias, mientras que los mecanísmos relacionados del endotelio se encuentra entre las bandas de frecuencias intermedias. Entonces surge la duda sobre los resultados que se obtendrán si se filtran los rangos de frecuencias asociados sólo al proceso de autorregulación.

Se desconoce a priori cómo es que aprenden los distintos modelos frente a señales alteradas y en distintos espectros de frecuencias y si acaso serán más eficientes que los modelos entrenados en todo el espectro de frecuencias.

\section{Objetivos del proyecto}
En esta sección se presenta el objetivo general y los objetivos específicos de la presente propuesta.

\subsection{Objetivo general}
Evaluar los modelos lineales y no lineales de aprendizaje de autorregulación del flujo sanguíneo cerebral basado en diferentes bandas de frecuencias para la comprensión del comportamiento de los modelos.

\subsection{Objetivos específicos}
Los objetivos establecidos para el presente trabajo son descritos a continuación
\begin{enumerate}
    \item Definir los modelos lineales y no lineales a implementar.
    \item Realizar el preprocesamiento de la señal de entrada.
    \item Entrenar los modelos lineales y no lineales con señales en distintas bandas de frecuencias.
    \item Entrenar los modelos lineales y no lineales con señales sintéticas.
    \item Establecer las conclusiones del trabajo.
\end{enumerate}

\section{Descripción de la solución}
\begin{comment}
En la presente sección se describe el estado del arte y las características de la solución. Se explicara cual es el propósito de la solución, y posteriormente los alcances y limitaciones establecidas.
\end{comment}

\subsection{Estado del arte}
Muchos de los métodos utilizados \cite{Elman1990, Schmidhuber1992b, Pearlmutter1989, Pearlmutter1995} sufren del desvanecimiento del gradiente. Para solventar el problema hay diversos métodos que lo evitan.
%% 13. S. E. Fahlman, "The recurrent cascade-correlation learning algorithm", in Advances in Neural Information Processing Systems 3, ed. R. P. Lippmann et al. (Morgan Kaufmann, San Mateo, 1991), pages 190-196.

% 14. R. J. Williams, "Complexity of exact gradient computation algorithms for recurrent neural networks" , Technical Report NU-CCS-89-27, Boston: Northeastern Univ., College of Computer Science, 1989.

%% 15. J. Schmidhuber, "A fixed size storage O(n^3) time complexity learning algorithm for fully recurrent continually running networks", Neural Computation, 4(2).243-248 (1992).

%% 16. B. A. Pearlmutter, "Learning state space trajectories in recurrent neural networks", Neural Computation, 1(2):263{269 (1989).

%% 17B. A. Pearlmutter, "Gradient calculations for dynamic recurrent neural networks: A survey", IEEE Transactions on Neural Networks, 6(5):1212{1228 (1995).


\subsubsection{Métodos de búsqueda global}
Los métodos de búsqueda global no utilizan el gradiente. Métodos como {\em simulated annealing} (SA), {\em multi-grid random search} \cite{Bengio1994} y {\em random weight guessing} \cite{Schmidhuber1996} han sido investigados. Se ha encontrado que los métodos de búsquedas globales funcionan bien en problemas que involucren dependencias a largo plazo y que además utilizan redes que contienen pocos parámetros y no precisan de alta precisión en sus calculos.

El método SA es un algoritmo de búsqueda meta-heurística para problemas de optimización global. La técnica está basada en el proceso de calentamiento de un metal, vidrio o cristal hasta su punto de fusión, para luego enfriarlo hasta obtener una estructura cristalina. El algoritmo \ref{alg:sa} decribe el procedimiento de la siguiente manera: dada una solución inicial se genera un conjunto de soluciones que son una perturbación de la solución actual, al evaluar cada perturbación se actualizará la solución actual en caso de que la perturbación sea mejor que la solución actual, y en caso contrario, existirá una probabilidad de que la solución actual sea actualizada de igual manera.

\scalebox{0.8}{\begin{algorithm}[H]
    \SetAlgoLined
	\DontPrintSemicolon
	\KwData{Temperatura $T$, constante de Boltzmann $k$, factor de reducción $c$}
	Seleccionar el mejor vector solución $x_{0}$ a optimizar\;
	\While{El criterio no se cumpla}{
		\While{Existan soluciones en el conjunto}{
			Seleccionar una solución $x_{0} + \Delta x$\;
			\uIf{$f(x_{0} + \Delta x) < f(x_{0})$}{
				$f_{new} = f(x_{0} + \Delta x)$; $x_{0} = x_{0} + \Delta x$\;
			}
			\Else{
				$\Delta f = f(x_{0} + \Delta x) - f(x_{0})$\;
				\uIf{$rand(0, 1) > \exp{(-\Delta f/kT}$)}{
					$f_{new} = f(x_{0} + \Delta x)$; $x_{0} = x_{0} + \Delta x$\;
				}\Else{
					$f_{new} = f(x_{0})$\;
				}
			}
			$f = f_{new}$; $T = cT$\;
		}
	}
	\caption{Simulated annealing}
	\label{alg:sa}
\end{algorithm}}



% [1998b] Hochreiter
% Y. Bengio, P. Simard, and P. Frasconi, "Learning long-term dependencies with gradient descent is dicult", IEEE Transactions on Neural Networks, 5(2):157{166 (1994).
%\subsubsection{(ii) Métodos que refuerzan el gradiente}
%Los valores más grandes del gradiente pueden ser reforzados por la optimización pseudo-Newton ponderada en el tiempoy la propagación discreta del errror \cite{Bengio1994}. Presentan problemas para almacenar información real de gran valor en el tiempo.

%\subsubsection{(iii) Métodos que operan en niveles mas altos}
%Anteriormente se ha propuesto un enfoque EM para la propagación del objetivo \cite{Bengio1993}. Este enfoque utiliza un número discreto de estados y, por lo tanto, tendrá problemas con valores continuos.
%Las técnicas de filtrado de Kallman se utilizan para el entrenamiento de redes recurrentes \cite{Puskorius1994}. Sin embargo, un factor de descuento derivado conduce a problemas de desvanecimiento del gradiente.
%Si un problema de retraso a largo plazo contiene regularidades locales, un sistema jerárquico chunket funciona bien \cite{Schmidhuber1992a}.

%\subsubsection{(iv) Métodos que utilizan arquitecturas especiales}
\subsubsection{El modelo de memoria a corto y largo plazo}
\citeA{Hochreiter1997a} introdujeron el modelo de memoria a corto y largo plazo ({\em Long short-term memory}, LSTM) como solución al problema del devanecimiento del gradiente. La red LSTM se basa en el bloque de memoria, que se componse de una o más celdas de memoria, una compuerta de entrada y una compuerta de salida. Las entradas son unidades multiplicativas con activación continua y son compartidas por todas las celdas de un mismo bloque de memoria. Cada celda contiene una unidad lineal con una conexión recurrente local llamada carrusel de error constante (CEC), se conocerá como estado de la celda a la activación del CEC.

Cada celda recibe una entrada ponderada por los pesos correspondientes a la capa anterior. La compuerta de entrada se encarga de de permitir o impedir el acceso de estos valores al CEC del interior de la celda. La compuerta de salida realiza una acción similar sobre la salida de la celda, tolerando o reprimiendo la difusión del estado del CEC al resto de la red.

Los bloques de memoria configuran una red LSTM, donde no se indican los sesgos de las distintas neuronas del modelo. La existencia de las conexiones con pesos $W^{y, u}$  determina la naturaleza de la red. Así, si se permite la existencia de esta conexión, la red LSTM se puede considerar como una máquina neuronal de estados de MEaly, si no se permite, la red LSTM puede considerarse como una máquina nerupna de estados de Moore. El estado de la red LSTM está formado por las activaciones de las compuertas, el CEC y las celdas de los bloques de memoria.

%%%%%%%%%%%%%%%%%%%%%%%%%%%%%%%%%%%%%%%%
% LiuShenxiu.pdf
% Conquering vanishing gradient: Tensor Tree LSTM on aspect-sentiment classification
% - Tree-structures LSTMs
% - Tensor Tree LSTM


%%%%%%%%%%%%%%%%%%%%%%%%%%%%%%%%%%%%%%%%
% Lipton2015
% Truncated backpropagation through time (TBPTT) is one solution to the exploding gradient problem for continuously running networks [Williams and Zipser, 1989]


%%%%%%%%%%%%%%%%%%%%%%%%%%%%%%%%%%%%%%%%
% Squartini2003a
% PREPROCESSING BASED SOLUTION FOR THE VANISHING
% Squartini2003b
% Attempting to reduce the vanishing gradient effect through a novel recurrent multiscale architecture
\subsubsection{Preprocesamiento de la señal}
\citeA{Squartini2003a} propone pre-procesar la señal de entrada a través de una descomposición wavelet, buscando separar la información a corto plazo de la información a largo plazo, y entrenando diferentes NN. Los resultados son combinados para alcanzar el objetivo final. Este enfoque simplifica el proceso de aprendizaje de la NN, evitando cambios relevantes en la arquitectura de la red y las técnicas de aprendizaje.

\subsection{Características de la solución}
La solución propóne un análisis práctico de la convergencia de las redes neuronales profundas mediante la aplicación de la regla de aprendizaje basada en el algoritmo {\em simulated annealing}. El análisis se realizará sobre conjuntos de datos de diferente indole, utilizados en otras investigaciones. Se comparará su desempeño frente a otras reglas de aprendizaje definidas en la literatura.

\subsection{Propósitos de la solución}
El propósito del presente trabajo es analizar la convergencia de las redes neuronales profundas, determinando el comportamiento de diferentes reglas de aprendizaje.

%El propósito del presente trabajo es comparar los resultados del aprendizaje de los métodos dinámicos de la autorregulación cerebral en los seres humanos, determinando las características del proceso en función de las bandas de frecuencias y ruido.

\subsection{Alcances o limitaciones de la solución}
Los alcances y limitaciones descritos para el trabajo son los siguientes
\begin{itemize}
	\item El estudio se plantea desde una perspectiva práctica, precisando conjuntos de datos acotados.

    \item Los datos que se utilizarán son utilizados por \citeA{Morse2016} en su publicación.

	\item Se estudiarán las reglas de aprendizaje definidas por el gradiente estocástico y {\em simulated annealing}.

	\item Las redes neuronales utilizarán la misma configuración para todos los experimentos salvo por la regla de aprendizaje.
\end{itemize}

\section{Metodología, herramientas y ambiente de desarrollo}
A continuación se especifica la metodología, herramientas y ambientes de desarrollo que se utilizarán para llebar a cabo el proyecto.

\subsection{Metodología a usar}
Considerando el aspecto investigativo del trabajo, se considera la utilización del método científico. Entre las actividades que componen la metodología, \cite{Sampieri2006} describe los siguientes pasos para desarrollar una investigación:

\begin{itemize}
    \item Formulación de la hipótesis: Los modelos lineales y no lineales que aprenden el comportamiento de la autorregulación del flujo sanguíneo cerebral se comportan distinto cuando se utilizan diferentes bandas de frecuencias.

    \item Marco teórico: Una revisión de la literatura donde se aborda el problema planteado, para situarse en el contexto actual de los problemas. Se describirán los modelos que permiten establecer relaciones entre las componentes de la autorregulación cerebral dinámica.

    \item Diseño de la solución: Se deberá diseñar el experimento para generar los modelos y preparar las señales para su evaluación. Diseñar y ejecutar el experimento provocando ruído en las señales utilizadas.

    % \item Análisis y verificación de los resultados: Los resultados se analizarán considerando las estadísticas que ofrece el experimento.
    \item Análisis y verificación de los resultados: Los resultados se analizarán considerando métodos estadísticos con un valor $p < 0.05$ que será considerado como diferencia significativa.

    \item Presentación de los resultados: Se presentarán tablas que describan los resultados obtenidos y que se consideren pertinentes.

    \item Conclusiones obtenidas en el desarrollo de la investigación.
\end{itemize}

\subsection{Herramientas de desarrollo}
Para el desarrollo y ejecución de los experimentos se utilizará un equipo con las siguientes características
\begin{table}[H]
    \centering
    \begin{tabular}{|l|l|}\hline
        Sistema Operativo   & Linux Mint 17.2 Cinnamon 64-bit\\\hline
        Procesador          & Intel Core i5-2450M CPU @ 2.50GHz x2\\\hline
        RAM                 & 7.7Gb\\\hline
        Almacenamiento      & 957.2Gb\\\hline
    \end{tabular}
\end{table}

El software que se utilizará es:
\begin{itemize}
    \item Software: Entorno para computación y gráficos estadísticos R.
    \item Herramienta ofimática: \LaTeX.
\end{itemize}

\subsection{Ambiente de desarrollo}
El desarrollo de la investigación se realizará en el Departamento de Ingeniería Informática de la Universidad de Santiago de Chile, el cual cuenta con una biblioteca y un laboratorio de computación con acceso a internet, para la recopilación de información y para el desarrollo del experimento. Además, se utilizará el hogar del autor, donde se ubica el equipo de desarrollo a utilizar.

\section{Plan de trabajo}
La planificación del trabajo de investigación, se indica en una carta Gantt que se puede apreciar en la Tabla \ref{tab:gantt}, donde se fija como fecha de inicio el día 12 de marzo de 2016 con fecha de término el día 30 de Junio de 2016. En un regimen de trabajo que considera periodos de 5 horas diarias de trabajo promedio, los 7 días de la semana, lo que equivale a un total de 640 horas de trabajo.

\begin{table}[H]
\centering
\caption{Carta Gantt}
\label{tab:gantt}
\scalebox{0.62} {
\begin{tabular}{lllccc}
\hline
\multicolumn{3}{c}{\textbf{\begin{tabular}[c]{@{}c@{}}NOMBRE DE\\ LA TAREA\end{tabular}}} & \textbf{\begin{tabular}[c]{@{}c@{}}FECHA DE\\ INICIO\end{tabular}} & \textbf{\begin{tabular}[c]{@{}c@{}}FECHA DE\\ TÉRMINO\end{tabular}} & \textbf{\begin{tabular}[c]{@{}c@{}}DURACIÓN\\ (DÍAS)\end{tabular}} \\ \hline
\multicolumn{3}{l}{PROYECTO DE TESIS}                                                     & 07-03-16                                                           &
24-06-16                                                            &                                                                    \\ \hline
     & INICIO		&							& 07-03-16	& 08-03-16	& 2\\
     &				& Reunión con el profesor	& 07-03-16	& 07-03-16	& 1\\
     &				& Inscripción del tema		& 08-03-16	& 08-03-16	& 1\\\hline
     & ESTADO		&							& 09-03-16	& 29-03-16	& 21\\
     & DEL ARTE		& Revisión bibliográfica	& 09-03-16	& 18-03-16	& 10\\
     &				& Reunión con el profesor	& 19-03-16	& 19-03-16	& 1\\
     &				& Marco teórico				& 20-03-16	& 29-03-16	& 10\\\hline
	 & DESARROLLO	&	& 30-03-16	& 24-04-16	& 26\\
     & LA SOLUCIÓN                             & Análisis                                           & 30-03-16                                                           & 02-04-16                                                            & 4                                                                  \\
     &                              & Diseño                                             & 03-04-16                                                           & 06-04-16                                                            & 4                                                                  \\
     &                              & Implementación                                     & 07-04-16                                                           & 16-04-16                                                            & 10                                                                 \\
     &                              & Pruebas                                            & 17-04-16                                                           & 19-04-16                                                            & 3                                                                  \\
     &                               & Correcciones                                       & 20-04-16                                                           & 22-04-16                                                            & 3                                                                  \\
     &                               & Reunión con el profesor                            & 23-04-16                                                           & 23-04-16                                                            & 1                                                                  \\
     &                               & Redacción del análisis, diseño e implementación    & 24-04-16                                                           & 24-04-16                                                            & 1                                                                  \\ \hline
     & EXPERIMENTOS                  &                                                    & 25-04-16                                                           & 25-05-16                                                            & 31                                                                 \\
     &                               & Calibración de parámetros                          & 25-04-16                                                           & 29-04-16                                                            & 5                                                                  \\
     &                               & Experimentación                                    & 30-04-16                                                           & 14-05-16                                                            & 15                                                                 \\
     &                               & Reunión con el profesor                            & 15-05-16                                                           & 15-05-16                                                            & 1                                                                  \\
     &                               & Redacción de la experimentación                    & 16-05-16                                                           & 25-05-16                                                            & 10                                                                 \\ \hline
     & ANÁLISIS DE RESULTADOS        &                                                    & 26-05-16                                                           & 08-06-16                                                            & 14                                                                 \\
     &                               & Análisis detallado                                 & 26-05-16                                                           & 29-05-16                                                            & 4                                                                  \\
     &                               & Interpretación de los resultados                   & 30-05-16                                                           & 02-06-16                                                            & 4                                                                  \\
     &                               & Reunión con el profesor                            & 03-06-16                                                           & 03-06-16                                                            & 1                                                                  \\
     &                               & Redación de los resultados                         & 04-06-16                                                           & 08-06-16                                                            & 5                                                                  \\ \hline
     & DOCUMENTO FINAL               &                                                    & 09-06-16                                                           & 24-06-16                                                            & 16                                                                 \\
     &                               & Redacción de la conclusión                         & 09-06-16                                                           & 10-06-16                                                            & 2                                                                  \\
     &                               & Revisión del documento                             & 11-06-16                                                           & 17-06-16                                                            & 7                                                                  \\
     &                               & Presentación al profesor                           & 18-06-16                                                           & 18-06-16                                                            & 1                                                                  \\ \hline
     & CORRECCIONES                  &                                                    & 19-06-16                                                           & 22-06-16                                                            & 4                                                                  \\
     &                               & Versión final                                      & 23-06-16                                                           & 23-06-16                                                            & 1                                                                  \\
     &                               & Entrega del documento                              & 24-06-16                                                           & 24-06-16                                                            & 1                                                                  \\ \hline
     &                               &                                                    & \multicolumn{1}{l}{}                                               & \multicolumn{1}{l}{}                                                & \multicolumn{1}{l}{}                                               \\
     &                               &                                                    & \multicolumn{1}{l}{}                                               & \multicolumn{1}{l}{}                                                & \multicolumn{1}{l}{}
\end{tabular}
}
\end{table}
%\end{comment}

%
%%%%%%%%%%%%%%%%%%%%%%%%%%%%
% Fin cosas importante
%%%%%%%%%%%%%%%%%%%%%%%%%%%


%%%%%%%%%%%%%%%%%%%%%%%%
% Bibliografía
%%%%%%%%%%%%%%%%%%%%%%%%
\addcontentsline{toc}{section}{Bibliografía}
\newpage
\bibliografia
\bibliographystyle{apacite}
\bibliography{referencias}
% \bibliographystyle{apa-good}
%%%%%%%%%%%%%%%%%%%%%%%%
% Fin bibliografía
%%%%%%%%%%%%%%%%%%%%%%%%


%%%%%%%%%%%%%%%%%%%%%%%%
% Apéndice
%%%%%%%%%%%%%%%%%%%%%%%%
%
% \appendix
% \clearpage
% \addappheadtotoc
% \appendixpage
%
\input{contenido/07-anexo}
%%%%%%%%%%%%%%%%%%%%%%%%
% Fin apéndice
%%%%%%%%%%%%%%%%%%%%%%%%
\end{document}
