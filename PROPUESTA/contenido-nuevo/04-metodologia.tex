\section{Metodología, herramientas y ambiente de desarrollo}
A continuación se especifica la metodología, herramientas y ambientes de desarrollo que se utilizarán para llebar a cabo el proyecto.

\subsection{Metodología a usar}
Considerando el aspecto investigativo del trabajo, se considera la utilización del método científico. Entre las actividades que componen la metodología, \cite{Sampieri2006} describe los siguientes pasos para desarrollar una investigación:

\begin{itemize}
    \item Formulación de la hipótesis: Los modelos lineales y no lineales que aprenden el comportamiento de la autorregulación del flujo sanguíneo cerebral se comportan distinto cuando se utilizan diferentes bandas de frecuencias.

    \item Marco teórico: Una revisión de la literatura donde se aborda el problema planteado, para situarse en el contexto actual de los problemas. Se describirán los modelos que permiten establecer relaciones entre las componentes de la autorregulación cerebral dinámica.

    \item Diseño de la solución: Se deberá diseñar el experimento para generar los modelos y preparar las señales para su evaluación. Diseñar y ejecutar el experimento provocando ruído en las señales utilizadas.

    % \item Análisis y verificación de los resultados: Los resultados se analizarán considerando las estadísticas que ofrece el experimento.
    \item Análisis y verificación de los resultados: Los resultados se analizarán considerando métodos estadísticos con un valor $p < 0.05$ que será considerado como diferencia significativa.

    \item Presentación de los resultados: Se presentarán tablas que describan los resultados obtenidos y que se consideren pertinentes.

    \item Conclusiones obtenidas en el desarrollo de la investigación.
\end{itemize}

\subsection{Herramientas de desarrollo}
Para el desarrollo y ejecución de los experimentos se utilizará un equipo con las siguientes características
\begin{table}[H]
    \centering
    \begin{tabular}{|l|l|}\hline
        Sistema Operativo   & Linux Mint 17.2 Cinnamon 64-bit\\\hline
        Procesador          & Intel Core i5-2450M CPU @ 2.50GHz x2\\\hline
        RAM                 & 7.7Gb\\\hline
        Almacenamiento      & 957.2Gb\\\hline
    \end{tabular}
\end{table}

El software que se utilizará es:
\begin{itemize}
    \item Software: Entorno para computación y gráficos estadísticos R.
    \item Herramienta ofimática: \LaTeX.
\end{itemize}

\subsection{Ambiente de desarrollo}
El desarrollo de la investigación se realizará en el Departamento de Ingeniería Informática de la Universidad de Santiago de Chile, el cual cuenta con una biblioteca y un laboratorio de computación con acceso a internet, para la recopilación de información y para el desarrollo del experimento. Además, se utilizará el hogar del autor, donde se ubica el equipo de desarrollo a utilizar.
